\documentclass[lineno]{biometrika}

\usepackage{amsmath}
%\usepackage{graphics}

%% Please use the following statements for
%% managing the text and math fonts for your papers:
%\usepackage{times}
%\usepackage[cmbold]{mathtime}
%\usepackage{bm}

\usepackage{newtxtext}
\usepackage[subscriptcorrection]{newtxmath}

% \usepackage{natbib}

\graphicspath{{./art/}}

\usepackage[plain,noend]{algorithm2e}

\makeatletter
\renewcommand{\algocf@captiontext}[2]{#1\algocf@typo. \AlCapFnt{}#2} % text of caption
\renewcommand{\AlTitleFnt}[1]{#1\unskip}% default definition
\def\@algocf@capt@plain{top}
\renewcommand{\algocf@makecaption}[2]{%
  \addtolength{\hsize}{\algomargin}%
  \sbox\@tempboxa{\algocf@captiontext{#1}{#2}}%
  \ifdim\wd\@tempboxa >\hsize%     % if caption is longer than a line
    \hskip .5\algomargin%
    \parbox[t]{\hsize}{\algocf@captiontext{#1}{#2}}% then caption is not centered
  \else%
    \global\@minipagefalse%
    \hbox to\hsize{\box\@tempboxa}% else caption is centered
  \fi%
  \addtolength{\hsize}{-\algomargin}%
}
\makeatother

%%% User-defined macros should be placed here, but keep them to a minimum.
\def\Bka{{\it Biometrika}}

\def\AIC{\textsc{aic}}
\def\T{{ \mathrm{\scriptscriptstyle T} }}
\def\v{{\varepsilon}}

% Notation
\usepackage{dsfont,enumitem, url}
\DeclareMathOperator{\E}{\mathbb{E}} % expectation
\newcommand{\Z}{\mathbb{Z}}
\newcommand{\R}{\mathbb{R}}
\newcommand{\N}{\mathbb{N}}
\newcommand{\C}{\mathbb{C}}
\newcommand{\blank}{\makebox[1ex]{\textbf{$\cdot$}}}
\newcommand\independent{\protect\mathpalette{\protect\independenT}{\perp}}
\def\independenT#1#2{\mathrel{\rlap{$#1#2$}\mkern2mu{#1#2}}}
\renewcommand{\phi}{\varphi}
\renewcommand{\epsilon}{\varepsilon}
\newcommand*\diff{\mathop{}\!\mathrm{d}}
\newcommand{\weakly}{\rightsquigarrow}
\newcommand\smallO{\textit{o}}
\newcommand\bigO{\textit{O}}
\newcommand{\midd}{\; \middle|\;}
\newcommand{\1}{\mathds{1}}
\usepackage{ifthen} %% Empirical process with default argument
\newcommand{\G}[2][n]{
{\ensuremath{\mathbb{G}_{#1}}{\left[#2\right]}}
}
\DeclareMathOperator*{\argmin}{\arg\!\min}
\DeclareMathOperator*{\argmax}{\arg\!\max}
\newcommand{\data}{\ensuremath{\mathcal{D}}}


%\addtolength\topmargin{35pt}
\DeclareMathOperator{\Thetabb}{\mathcal{C}}

% \usepackage[colorlinks,allcolors=blue]{hyperref}

\begin{document}

\jname{Biometrika}
%% The year, volume, and number are determined on publication
\jyear{2025}
\jvol{112}
\jnum{1}
\cyear{2025}
%% The \doi{...} and \accessdate commands are used by the production team
%\doi{10.1093/biomet/asm023}
\accessdate{Advance Access publication on 14 February 2025}

%% These dates are usually set by the production team
\received{2 January 2024}
\revised{3 February 2025}

%% The left and right page headers are defined here:
\markboth{Munch and Gerds}{The joint survival super learner}

%% Here are the title, author names and addresses
\title{The joint survival super learner: A super learner for right-censored
  data}

\author{A. MUNCH} %
\affil{Section of Biostatistics, University of
  Copenhagen \email{a.munch@sund.ku.dk}}

\author{\and T. A. GERDS}
\affil{Section of Biostatistics, University of Copenhagen}

\maketitle

\begin{abstract}
  Risk prediction models are widely used to guide real-world
  decision-making in areas such as healthcare and economics, and they
  also play a key role in estimating nuisance parameters in
  semiparametric inference. The super learner is a machine learning
  framework that combines a library of prediction algorithms into a
  meta-learner using cross-validated loss. In the context of
  right-censored data, careful consideration must be given to both the
  choice of loss function and the estimation of expected loss.
  Moreover, estimators such as inverse probability of censoring
  weighting (IPCW) require accurate modeling and estimates of the
  censoring distribution. We propose a novel approach to super
  learning for survival analysis that jointly evaluates candidate
  learners for both the event-time distribution and the censoring
  distribution. Our method imposes no restrictions on the algorithms
  included in the library, accommodates competing risks, and does not
  rely on a single pre-specified estimator of the censoring
  distribution. We establish theoretical guarantees for our proposed
  method, including a finite-sample oracle inequality. In a simulation
  study, our super learner was able to better account for different
  censoring mechanisms than existing methods. We demonstrate the
  practical utility of our method using prostate cancer data.
\end{abstract}

\begin{keywords}
competing risks; cross-validation; loss based estimation; right-censored data; super learner
\end{keywords}

\section{Introduction}
\label{sec:introduction}

Accurately predicting risk from time-to-event data is a central
challenge in diverse research fields, including epidemiology,
economics, and weather forecasting, with applications in clinical
decision making and policy interventions. For instance, in prostate
cancer management, clinicians often need to estimate a patient’s risk
of disease progression and mortality over time to make informed
decisions about treatment strategies such as active surveillance
versus immediate intervention. Reliable time-to-event risk predictions
can help tailor care to individual patients, avoid overtreatment, and
allocate healthcare resources more effectively. Super learning
\citep{van2007super}, also known as ensemble learning or stacked
regression \citep{wolpert1992stacked,breiman1996stacked}, provides a
powerful approach to this problem by combining multiple candidate
prediction models to reduce the risk of bias incurred by a single
potentially mispecified model. In survival analysis, a super learner
may combine a stack of Cox regression models with a stack of random
survival forests \citep[][Section 8.4]{gerds2021medical}. Such a
strategy has recently produced KDpredict
(\url{https://kdpredict.com/}) a model which jointly predicts the
risks of kidney failure and all-cause mortality at multiple time
horizons based on different sets of covariates
\citep{liu2024predicting}. To evaluate the prediction performance of
the learners, the super learner behind KDpredict uses inverse
probability of censoring weighting (IPCW), where the censoring
distribution is estimated under the restrictive assumption that it
does not depend on the covariates. This is a potential source of bias
which is difficult to overcome with the currently available methods.

In this paper, we propose the joint survival super learner, a new
super learner designed to handle the specific challenges of ensemble
learning with right-censored data. The joint survival super learner
simultaneously learns prediction models for the event-time and
censoring distributions. The joint survival super learner is based on
the simple idea of an artificial competing risks model, in which
censoring is included as a state of its own. Candidate learners for the
event-time and censoring hazard functions are then assessed based on
how well they predict the state occupation probabilities of the
artificial competing risks model over time, given baseline
covariate information. Our estimation framework allows for competing
risks, avoids restrictive assumptions on the censoring distribution,
and is also fully flexible with respect to the choice of learners. The
latter is in contrast to other proposals which restrict the library of
learners to specific model classes
\citep{polley2011-sl-cens,golmakani2020super}, see
Section~\ref{sec:super-learning}.

To analyse the theoretical properties of the joint survival super
learner, we focus on the discrete super learner, which selects the
model in the library with the best estimated performance
\citep{van2007super}. We provide theoretical guarantees for the
performance of the joint survival super learner, and in particular
show that the discrete joint survival super learner is consistent
under natural conditions and satisfies a finite-sample oracle
inequality. We demonstrate how to construct a library using common
survival models and how to obtain risk predictions from the resulting
ensemble.

The rest of the paper is organized as follows. We introduce our
notation and framework in Section~\ref{sec:framework}.
Section~\ref{sec:super-learning} introduces loss-based super learning
and presents existing super learners for right-censored data. In
Section~\ref{sec:super-learner-simple} we define the joint survival
super learner, while Section~\ref{sec:theor-results-prop} provides
theoretical guarantees. Section~\ref{sec:numer-exper} reports the
results of numerical experiments, and Section~\ref{sec:real-data-appl}
illustrates the method on prostate cancer data. We conclude with a
discussion in Section~\ref{sec:discussion}. Proofs are collected in
the Appendix. Code and an implementation of the joint
survival super learner are available at
\url{https://github.com/amnudn/joint-survival-super-learner}.

\section{Notation and framework}
\label{sec:framework}

In a competing risks framework \citep{andersen2012statistical}, let \(
T\) be a time to event variable, \(D\in\{1,2\}\) the cause of the
event, and $X \in \mathcal{X}$ a vector of baseline covariates taking
values in a bounded subset \( \mathcal{X} \subset \R^p \), \( p\in\N
\). Let $\tau< \infty$ be a fixed prediction horizon. We use \(
\mathcal{Q} \) to denote the collection of all probability measures on
\( [0,\tau] \times \{1,2\}\times \mathcal{X} \) such that \( (T, D, X)
\sim Q \) for some unknown \( Q \in \mathcal{Q} \). For
\(j\in\{1,2\}\), the cause-specific conditional cumulative hazard
functions \( \Lambda_{j} \colon [0, \tau] \times \mathcal{X}
\rightarrow \R_+ \) are defined as
\begin{equation*}
  % \label{eq:cum-haz}
  \Lambda_{j}(t \mid x) = \int_0^t\frac{  Q(T \in \diff s, D=j \mid X=x )}{Q(T \geq s \mid X=x )}.
\end{equation*} For ease of presentation we assume throughout that the
map \( t\mapsto \Lambda_j(t \mid x) \) is continuous for all \( x \)
and \( j \), however, 
all technical arguments extend naturally to the general case \citep{andersen2012statistical}.
The event-free survival function conditional on covariates is
\begin{equation}
  \label{eq:surv-def}
  S(t \mid x)=\exp\left\{-\Lambda_{1}(t \mid x)-\Lambda_{2}(t \mid x)\right\}.
\end{equation}
Let \( \mathcal{M}_{\tau}\) denote the space of all conditional
cumulative hazard functions on \( [0,\tau] \times\mathcal{X}\). Any
distribution \( Q \in \mathcal{Q} \) can be characterized by
\begin{equation*}
  \label{eq:parametrizeQ}
  \begin{split}
    Q(\diff t,j,\diff x)=& \left\{S(t- \mid x)\Lambda_1(\diff t \mid x)H(\diff x)\right\}^{\1{\{j=1\}}}\\
                         &  \left\{S(t- \mid x)\Lambda_2(\diff t \mid x)H(\diff x)\right\}^{\1{\{j=2\}}},
  \end{split}
\end{equation*}
where \(\Lambda_{j} \in \mathcal{M}_{\tau}\) for \(j=1,2\) and \(H\) is the marginal
distribution of the covariates.

We consider the right-censored setting in which we observe \(O =
(\tilde{T},\tilde D, X)\), where $\tilde T = \min(T,C)$ for a
right-censoring time \(C\), $\Delta = \1{\{T \leq C\}}$, and \(\tilde
D=\Delta D\). Let \(\mathcal{P}\) denote a set of probability measures
on the sample space \(\mathcal{O} = [0, \tau] \times \{0, 1, 2\}
\times \mathcal{X}\) such that \(O \sim P \) for some unknown \(P\in
\mathcal{P}\). We assume that the event times and the censoring times
are conditionally independent given covariates, \( T \independent C
\mid X \). This implies that any distribution \( P \in \mathcal{P} \)
is characterized by a distribution \( Q \in \mathcal{Q} \) and a
conditional cumulative hazard function for \( C \) given \( X \)
\citep[c.f.,][]{begun1983information,gill1997coarsening}. We use
\(\Gamma\in\mathcal{M}_{\tau}\) to denote the cumulative hazard
function of the conditional censoring distribution given
covariates. For ease of presentation we assume that \(t\mapsto
\Gamma(t \mid x) \) is continuous for all \( x \). We let
\((t,x)\mapsto G(t \mid x)=\exp\left\{-\Gamma(t \mid x)\right\}\)
denote the survival function of the conditional censoring
distribution. The distribution \( P \) is characterized by
\begin{equation}\label{eq:parametrizeP}
  \begin{split}
    P(\diff t, j, \diff x) =& \left\{G(t- \mid x)S(t- \mid x)\Lambda_1(\diff t \mid x)H(\diff x)\right\}^{\1{{\{j=1\}}}}\\
                            & \left\{G(t- \mid x)S(t- \mid x)\Lambda_2(\diff t \mid x)H(\diff x)\right\}^{\1{{\{j=2\}}}}\\
                            & \left\{G(t- \mid x)S(t- \mid x)\Gamma(\diff t \mid x)H(\diff x)\right\}^{\1{{\{j=0\}}}}\\
    = & \left\{G(t- \mid x)Q(\diff t,j,\diff x)\right\}^{\1{{\{j\ne 0\}}}}\\    
                            & \left\{G(t- \mid x)S(t- \mid x)\Gamma(\diff t \mid x)H(\diff x)\right\}^{\1{{\{j=0\}}}}.
  \end{split}
\end{equation}
Hence, we may write
\( \mathcal{P} = \{ P_{Q, \Gamma} : Q \in \mathcal{Q}, \Gamma \in
\mathcal{G} \} \) for some \( \mathcal{G} \subset \mathcal{M}_{\tau} \). We
also have \(H\)-almost everywhere
\begin{equation*}
P(\tilde T>t \mid X=x) = S(t \mid x)G(t \mid x) = \exp\left\{-\Lambda_{1}(t \mid x)-\Lambda_{2}(t \mid x)-\Gamma(t \mid x) \right\}.
\end{equation*} We assume that there exists \(\kappa<\infty\) such
that \(\Lambda_{j}(\tau- \mid x)<\kappa \), for \(j\in\{1,2\}\), and
\(\Gamma(\tau- \mid x)<\kappa\) for almost all \(x\in\mathcal
X\). This implies that \(G(\tau- \mid x)\) is bounded away
from zero for almost all \(x\in\mathcal X\).  Under these assumptions,
the conditional cumulative hazard functions \(\Lambda_{j}\) and
\(\Gamma\) can be identified from \(P\) by
\begin{align}
  \Lambda_{j}(t \mid x) &= \int_0^t\frac{  P(\tilde T \in \diff s, \tilde D=j \mid X=x )}{P(\tilde T \geq s \mid X=x )}, \label{eq:lambdaj}\\
  \Gamma(t \mid x) &= \int_0^t\frac{  P(\tilde T \in \diff s, \tilde D=0 \mid X=x )}{P(\tilde T \geq s \mid X=x )}\label{eq:gamma}.
\end{align}
Thus, we can consider $\Lambda_j$ and \(\Gamma\) as operators which map from
\( \mathcal{P} \) to \(\mathcal M_{\tau}\).

\section{Loss-based super learning}
\label{sec:super-learning}

Loss-based super learning requires a library of candidate models (or
\textit{learners}), a cross-validation algorithm, and a loss function
for evaluating predictive performance on hold-out samples. Let
\( \data_n=\{O_i\}_{i=1}^n \in \mathcal{O}^n \) be a data set of
i.i.d.\ observations from \( P \in \mathcal{P} \), and $\mathcal{A}$ a
collection of candidate learners. Let \(\Theta\) be the parameter
space, which in our case is a class of functions representing
different models. Each learner \(a \in \mathcal{A}\) is a map
\( a \colon \mathcal{O}^n \rightarrow \Theta \) which takes a data set
as input and returns an estimate $a(\data_n) \in \Theta$. Let
\(L\colon \Theta \times \mathcal{O} \rightarrow \R_+\) be a loss
function, representing the performance of the model
$\theta \in \Theta$ at the observation \( O \in \mathcal{O} \), where
lower values mean better performance.

The expected loss of a learner is estimated by splitting the data set
$\data_n$ into $K$ disjoint approximately equally sized subsets
\(\data_n^1, \data_n^2, \dots, \data_n^K \) and then calculating the
cross-validated loss
\begin{equation}
  \label{eq:cv-risk-est}
  \hat{R}_n(a; L) =
  \frac{1}{K}\sum_{k=1}^{K}
  \frac{1}{| \data_n^{k} |}\sum_{O_i \in \data_n^{k}}
  L
  {
    \left(
      a{ (\data_n^{-k})}
      , O_i
    \right)
  },
  \quad \text{with} \quad
  \data_n^{-k} = \data_n \setminus \data_n^{k}.
\end{equation}
The subset \(\data_n^{-k}\) is referred to as the \(k\)'th training
sample, while \(\data_n^{k}\) is referred to as the \(k\)'th test or
hold-out sample.
The discrete super learner is defined as
\begin{equation*}
\hat{a}_n = \argmin_{a\in\mathcal A}\hat{R}_n(a; L),
\end{equation*}
and depends on both the library of learners and the specific
partitioning of the data into cross-validation folds
\( \data_n^1, \dots, \data_n^K \).

When designing a super learner for right-censored data, particular
care must be taken in the choice of loss function and in the
estimation of the expected loss. A commonly used loss function for
right-censored data is the partial log-likelihood loss
\citep[e.g.,][]{li2016regularized,yao2017deep,lee2018deephit,katzman2018deepsurv,gensheimer2019scalable,lee2021boosted,kvamme2021continuous}.
This loss function is also recommended for super learning with
right-censored data by \cite{polley2011-sl-cens}, under the assumption
that data are observed in discrete time. However, the partial
log-likelihood loss does not work well as a measure of performance in
hold-out samples observed in continuous time. The is because it
assigns an infinite value to any learner that predicts piecewise
constant cumulative hazard functions, if the test set contains event
times that were not observed in the training set. This problem occurs
with prominent survival learners including the Kaplan-Meier estimator,
random survival forests, and semi-parametric Cox regression models,
and these learners cannot be included in the library of the super
learner proposed by \cite{polley2011-sl-cens}. When a proportional
hazards model is assumed, the baseline hazard function can be profiled
out of the likelihood \citep{cox1972regression}. The cross-validated
partial log-likelihood loss \citep{verweij1993cross} has therefore
been suggested as a loss function for super learning by
\cite{golmakani2020super}. This choice of loss function restricts the
library of learners to include only Cox proportional hazards models,
and hence excludes many learners such as, e.g., random survival
forests, additive hazards models, and accelerated failure time models.

Alternative approaches for super learning with right-censored data use
an inverse probability of censoring weighted (IPCW) loss function
\citep{graf1999assessment,van2003unicv,molinaro2004tree,keles2004asymptotically,hothorn2006survival,gerds2006consistent,gonzalez2021stacked},
censoring unbiased transformations
\citep{fan1996local,steingrimsson2019censoring}, or pseudo-values
\citep{andersen2003generalised,mogensen2013random,sachs2019ensemble}.
All these methods rely on an estimator of the censoring distribution,
and their drawback is that this estimator has to be pre-specified.
Recent work by
\cite{han2021inverse} and \cite{westling2021inference} 
circumvents the need to pre-specify a censoring model by iterating
between estimation of the outcome and censoring models. However, this
iterative procedure is in general not guaranteed to converge to the
true data-generating mechanism
\citep[][Appendix~A.4]{munch2024thesis}.


\section{The joint survival super learner}
\label{sec:super-learner-simple}

\subsection{An artificial competing risks model}
\label{sec:an-artif-comp}


The main idea of the joint survival super learner is to jointly use
learners for \( \Lambda_1 \), \( \Lambda_2 \), and \( \Gamma \), and
the relations in equation~(\ref{eq:parametrizeP}), to learn a feature
of the observed data distribution \( P \). The discrete joint survival
super learner ranks a tuple of learners for \( (\Lambda_1, \Lambda_2, \Gamma) \) based
on how well they jointly model the observed data. To formally
introduce the joint survival super learner, we define the process
\begin{equation*}
  \eta(t) = \1\{\tilde{T} \leq t, \tilde D=1\} + 2\,\1\{\tilde{T} \leq t, \tilde
  D=2\} - \1\{\tilde{T} \leq t, \tilde D=0\},
  \quad \text{for} \quad t \in [0, \tau],
\end{equation*}
which takes values in \( \{-1,0,1,2\}\). The four values
represent four mutually exclusive states. Specifically, value
\( 0 \) represents the state where the individual is still
event-free and uncensored, value \( 1\) the state where the
event of interest has occurred, value \( 2\) the state where a
competing risk has occurred, and value \( -1\) the state where
the observation is right-censored. The state occupation
probabilities given baseline covariates \( X \) are given by
the function
\begin{equation}
  \label{eq:F-def}
  F(t, l, x) = P(\eta(t) = l \mid X=x),
\end{equation}
for all \( t \in [0,\tau] \), \( l \in \{-1,0,1,2\} \), and
\( x \in \mathcal{X} \).

The joint survival super learner is a super learner for the
function-valued parameter $\theta(P) = F$ which is identified through
equation~(\ref{eq:F-def}). Under conditional independent censoring,
each tuple $(\Lambda_{1}, \Lambda_{2}, \Gamma, H)$ characterizes a
distribution \(P\in\mathcal P\), c.f.\
equation~\eqref{eq:parametrizeP}, which in turn determines
\( (F, H) \). Hence, a learner for \( F \) can be constructed from
learners for \( \Lambda_1 \), \( \Lambda_2 \), and $\Gamma$ as
follows:
\begin{equation}\label{eq:transition}
  \begin{split}
    F(t, 0, x)
    &
      = P(\tilde{T} > t \mid X= x)
      =
      \exp{{\{-\Lambda_{1}(t \mid x)-\Lambda_{2}(t \mid x) - \Gamma(t \mid x)\}
      }},
    \\
    F(t, 1, x)
    &
      = P(\tilde{T} \leq t, \tilde{D}=1 \mid X=x)
      = \int_0^t F(s-, 0, x)  \Lambda_{1}(\diff s \mid x),
    \\
    F(t, 2, x)
    &
      = P(\tilde{T} \leq t, \tilde{D}=2 \mid X=x)
      = \int_0^t  F(s-, 0, x)  \Lambda_{2}(\diff s \mid x),
    \\
    F(t, -1, x)
    &
      = P(\tilde{T} \leq t, \tilde{D}=0 \mid X=x)
      = \int_0^t F(s-, 0, x)  \Gamma(\diff s \mid x).
  \end{split}
\end{equation}
Equation~\eqref{eq:transition} implies that a library for the joint survival super learner can by build from three libraries of learners:
\(\mathcal{A}_1\), \( \mathcal{A}_2 \), and \( \mathcal{B} \),
where \(\mathcal{A}_1\) and \( \mathcal{A}_2\) contain
learners for the conditional cause-specific cumulative hazard
functions \(\Lambda_1\) and \( \Lambda_2\), respectively, and
\(\mathcal{B}\) contains learners for the conditional
cumulative hazard function of the censoring distribution.
Taking the Cartesian product of these libraries, we obtain a library
$\mathcal{F}$ of learners for \( F \):
\begin{equation}
  \label{eq:jssl-lib-def}
  \mathcal{F}(\mathcal{A}_1, \mathcal{A}_2, \mathcal{B})
  = \{ \phi_{a_1,a_2, b} : a_1 \in \mathcal{A}_1, a_2 \in \mathcal{A}_2, b \in \mathcal{B}\},
\end{equation}
where in correspondence with the relations in equation
\eqref{eq:transition},
\begin{align*}
    \phi_{a_1,a_2, b}(\data_n)(t,0,x)
  &= \exp{\{-a_1(\data_n)(s \mid x)-a_2(\data_n)(s \mid x) - b(\data_n)(s \mid
    x)\} },
  \\
  \phi_{a_1,a_2, b}(\data_n)(t,1,x)
  &= \int_0^t
    \phi_{a_1,a_2, b}(\data_n)(s-,0,x)  a_1(\data_n)(\diff s \mid x),
  \\
  \phi_{a_1,a_2, b}(\data_n)(t,2,x)
  &= \int_0^t\phi_{a_1,a_2, b}(\data_n)(s-,0,x)  a_2(\data_n)(\diff s \mid x),
  \\
  \phi_{a_1,a_2, b}(\data_n)(t,-1,x)
  &= \int_0^t \phi_{a_1,a_2, b}(\data_n)(s-,0,x)  b(\data_n)(\diff s \mid x).
\end{align*} Notably, the libraries \( \mathcal{A}_1 \), \(
\mathcal{A}_2 \), and \( \mathcal{B} \) can be constructed using
standard software for survival analysis. In
\texttt{R}, for instance, we can construct Cox models as
learners using the \texttt{survival}-package
\citep{survival-package}, and we can construct random survival
forests as learners using the \texttt{randomForestSRC}-package \citep{randomForestSRC}.

To evaluate how well a function \( F \) predicts the
process $\eta$ we use the integrated Brier score \citep{graf1999assessment}
\( \bar B_\tau( F,O) = \int_0^{\tau} B_t(F,O) \diff t \),
where \( B_t \) is the Brier score
\citep{brier1950verification} at time \( t \in [0, \tau] \),
\begin{equation*}
  B_t(F,O) = \sum_{l=-1}^{2}
  \left(
      F(t,l,X) - \1{\{\eta(t)=l\}}
  \right)^2.
\end{equation*}
The Brier score is here the squared prediction error across
all of the four states. Based on a split of a data set \(\data_n\)
into $K$ disjoint approximately equally sized subsets (c.f., Section
\ref{sec:super-learning}), each learner \( \phi_{a_1, a_2, b} \) in
the library
\( \mathcal{F}(\mathcal{A}_1, \mathcal{A}_2, \mathcal{B}) \) is
evaluated using the cross-validated loss,
\begin{equation*}
  \hat{R}_{n}(\phi_{a_1,a_2,b} ; \bar{B}_{\tau}) =
  \frac{1}{K}\sum_{k=1}^{K}
  \frac{1}{| \data_n^{k} |}\sum_{O_i \in \data_n^{k}}
  \bar B_\tau
  {
    \left(
      \phi_{a_1,a_2,b}{ (\data_n^{-k})}
      , O_i
    \right)
  },
\end{equation*}
and the discrete joint survival super learner is 
\begin{align*}\label{eq:discrete-state-learner}
  \hat{\phi}_n
  &=  \argmin_{(a_1,a_2,b)\in \mathcal{A}_1\times\mathcal{A}_2\times\mathcal{B}}
    \hat{R}_{n}(\phi_{a_1,a_2,b} ; \bar{B}_{\tau}).
\end{align*}




\subsection{Obtaining risk predictions}
\label{sec:use-cases-state}

The joint survival super learner estimates the function \( F \) which
depends on the censoring distribution and is therefore typically not
of direct interest in itself. For instance, in the prostate cancer
example we consider in Sections~\ref{sec:numer-exper}
and~\ref{sec:real-data-appl}, the value \( F(t, l, x) \) denotes the
conditional probability that a patient with baseline characteristics
\( X=x \) will, before time point \( t \): have had tumour recurrence
before leaving the study ($l=1$); have died without tumour recurrence
before leaving the study ($l=2$); have left the study (state $l=-1$);
or be alive and part of the study without tumour recurrence ($l=0$).
The reference to still being part of the study is irrelevant from a
new patient's perspective. We here demonstrate how clinically relevant
risk predictions can instead be obtained from the joint survival super
learner.

We recall that we work under the assumption of conditional independent
censoring and positivity, as introduced in
Section~\ref{sec:framework}. Under these assumptions it follows by
equations~(\ref{eq:lambdaj}) and~(\ref{eq:gamma}) and the definition
of \( F \) that
\begin{equation}
  \label{eq:7}
  \Lambda_j(t , x) 
  = \int_0^t  \frac{F(\diff s, j, x )}{F(s-, 0, x )},
  \quad j \in \{1,2\}.
\end{equation}
Cause-specific risk predictions can be obtained from \( \Lambda_1 \)
and $\Lambda_2$ using the formula
\citep[e.g.,][]{benichou1990estimates, ozenne2017riskregression},
\begin{equation}
  \label{eq:cs-risk-def}
  Q(T \leq t, D = j \mid X=x) =
  \int_0^t \exp\left\{-\Lambda_{1}(u \mid x)-\Lambda_{2}(u
    \mid x)\right\}  \Lambda_j(\diff u \mid x),
  \quad j \in \{1,2\}.
\end{equation}
Hence, given the joint survival super learner's estimate of \( F \) we
can use equation~\eqref{eq:7} to obtain estimates of the
cause-specific cumulative hazard functions $\Lambda_j$, which can in
turn be used to obtain estimates of the cause-specific risks through
equations~\eqref{eq:surv-def}. For instance, in the prostate cancer
example, this expression will provide the conditional probability that
a patient with a certain set of baseline characteristics will, before
time point \( t \), have had tumour recurrence, have died without tumour
recurrence, or be alive without tumour recurrence.

We have suggested to implement the joint survival super learner by building a library
using learners of the cause-specific cumulative hazard functions,
$\Lambda_j$, and the cumulative hazard function for censoring,
$\Gamma$. With this implementation we can directly input the highest
ranked tuple of cause-specific hazard functions
$(\Lambda_1, \Lambda_2)$ provided by the joint survival super learner as input to
equation~\eqref{eq:cs-risk-def}.

\section{Theoretical guarantees}
\label{sec:theor-results-prop}

The use of cross-validation underlying the joint survival super
learner is an intuitively reasonable procedure for fair model
selection without overfitting. In this section, we follow the works of
\cite{van2003unicv} and \cite{vaart2006oracle} to provide a
theoretical justification for this practice in the form of a
finite-sample oracle inequality. We begin by demonstrating that
minimizing the integrated Brier score, as defined in
Section~\ref{sec:super-learner-simple}, is statistically meaningful,
in that perfect minimisation recovers the parameter of the
data-generating distribution. Together with our finite-sample oracle
inequality (Proposition~\ref{prop:oracle-prop} below), this implies
that the joint survival super learner is consistent when it is based
on a library that includes a consistent learner. Another consequence
of our finite-sample oracle inequality is that the joint survival
super learner converges at (nearly) the optimal rate achievable within
the library. This statement is made precise in
Corollary~\ref{cor:asymp-cons} and the following discussion. Proofs
are deferred to the Appendix.

A sensible loss function should attain the minimal expected value at
the parameter corresponding to the data-generating distribution. Loss
functions with this property are known as proper scoring rules, and as
strictly proper scoring rules if the minimize is unique
\citep{gneiting2007strictly}. Absence of properness makes it unclear
why minimizing the (estimated) expected loss is interesting.
Proposition~\ref{prop:stric-prop} is a formal statement of the fact
that the integrated Brier score, as defined in our setting (c.f.,
Section~\ref{sec:super-learner-simple}), is a strictly proper scoring
rule. To state this result, recall that the function \(F\) implicitly
depends on the data-generating probability measure \(P\in\mathcal P\)
but that this was suppressed in the notation. We now make this
dependence explicit by writing \(F_P\) for the function determined by
a given \(P \in\mathcal{P}\) in accordance with equation
equation~(\ref{eq:F-def}). In the following we let
\( \mathcal{H}_{\mathcal{P}} = \{F_P : P \in \mathcal{P}\} \).

\begin{proposition}
  \label{prop:stric-prop}
  If \(P \in\mathcal{P}\) then
  \begin{equation*}
    F_P = \argmin_{F \in \mathcal{H}_{\mathcal{P}}}
    \E_P{[\bar{B}_\tau(F, O)]}
    ,
  \end{equation*}
  for all \( l \in \{-1, 0, 1, 2 \} \), almost all
  \( t \in [0,\tau] \), and \( P \)-almost all
  \( x \in \mathcal{X} \).
\end{proposition}


To evaluate the performance of the joint survival super learner we
might benchmark it against the data-generating \( F_P \), as this has
smallest expected loss by Proposition~\ref{prop:stric-prop}. A more
nuanced comparison is to benchmark it against the best learner
available given the library and the training data. This is the
so-called oracle learner, formally defined as
\begin{equation*}
  \tilde{\phi}_n
  =  \argmin_{\phi \in \mathcal{F}(\mathcal{A}_1, \mathcal{A}_2, \mathcal{B}) }
  \tilde{R}_{n}(\phi ; \bar{B}_{\tau}),
  \quad \text{with} \quad 
  \tilde{R}_n(\phi; \bar{B}_{\tau})=
  \frac{1}{K}\sum_{k=1}^{K} 
  \E_P{
    \left[
      \bar{B}_{\tau}
      {
        \left(
          \phi{ (\data_n^{-k})}
          , O
        \right)
      } 
      \midd  \data_n^{-k}
    \right]}
  ,
\end{equation*}
where % \(  \mathcal{F}(\mathcal{A}_1, \mathcal{A}_2, \mathcal{B}) \)
% was defined in equation~\eqref{eq:jssl-lib-def}, and 
we use \( \E_P \) to denote expectation under the distribution \( P \)
for a new observation \( O \) independent of \( \data_n^{-k} \). Like
the joint survival super learner, the oracle learner depends on the library of learners and on the
specific data partitions, but unlike the joint survival super learner, it also depends on the
unknown data-generating distribution.

In the following, we equip the space \( \mathcal{H}_{\mathcal{P}} \)
with the norm
\begin{equation}
  \label{eq:norm}
  \| F \|_{P} = 
  \left\{
    \sum_{l=-1}^{2}
    \int_0^{\tau} \E_P{\left[ F(t, l, X)^2 \right]} \diff t
  \right\}^{1/2}.
\end{equation}
This norm is equal to the excess risk
\( \E_P{[\bar{B}_\tau(F, O)]} - \E_P{[\bar{B}_\tau(F_P, O)]} \) by
Lemma~\ref{lemma:norm} in the Appendix, and is thus a natural
performance measure. For simplicity of presentation we take \( n \)
and the data partitions to be such that \( |\data_n^{-k}| = n/K \)
with \( K \) fixed. We will allow the number of learners to grow with
\( n \) and write
\( \mathcal{F}_n=\mathcal{F}(\mathcal{A}_{1,n}, \mathcal{A}_{2,n},
\mathcal{B}_n)\) as short-hand notation emphasing the dependence on
the sample size.  We now state a finite-sample inequality that bounds
the performance of the joint survival super learner relative to that of the oracle learner.

\begin{proposition}
  \label{prop:oracle-prop}
  For all \(P\in\mathcal{P}\), \( n \in \N \), \( k \in \{1, \dots, K\} \),
  and $\delta>0$,
  \begin{align*}
    \frac{1}{K}\sum_{k=1}^{K} \E_{P}{\left[ \Vert \hat{\phi}_n(\data_n^{-k}) - F_P \Vert_{P}^2 \right]}
    & \leq (1 + 2\delta)
      \frac{1}{K}\sum_{k=1}^{K} \E_{P}{\left[ \Vert \tilde{\phi}_n(\data_n^{-k}) - F_P \Vert_{P}^2 \right]}
    \\
    & \quad
      + (1+ \delta) 16   K \tau
      \left(
      13 + \frac{12}{\delta}
      \right)
      \frac{\log(1 + |\mathcal{F}_n|)}{n}.
  \end{align*}
\end{proposition}

The expectations in Proposition~\ref{prop:oracle-prop} reflect a mild
abuse of notation, in that they are formally taken with respect to the
product measure \( P^{n} \) for the whole data set \( \data_n \). This
means that we are quantifying the average performance of the joint survival super learner across
average training data. As for many finite-sample oracle inequalities,
this result is of little direct practical utility because the right
hand-side depends on data-dependent, unknown quantities. However, it
does quantify how the number of folds, the time horizon, and the
number of learners in the library can be expected to influence the
performance. The result has the following asymptotic consequences.

\begin{corollary}
  \label{cor:asymp-cons}
  Assume that \( |\mathcal{F}_n| = \bigO(n^q)\), for some
  \( q \in \N \) and that there exists a sequence
  \( \phi_n \in \mathcal{F}_n \), \( n \in \N \), such that
  \(  \E_{P}{\left[ \Vert
      \phi_n(\data_n^{-k}) - F_P \Vert_{P}^2 \right]} = C_P +
  \bigO(n^{-\alpha}) \), for some \( \alpha\leq 1 \) and
  \( C_P \geq 0 \).
  \begin{enumerate}[label=(\alph*)]
  \item\label{item:1} If $\alpha=1$ then
    \(\frac{1}{K}\sum_{k=1}^{K} \E_{P}{\left[ \Vert
        \hat{\phi}_n(\data_n^{-k}) - F_P \Vert_{P}^2 \right]} = C_P +
    \bigO(\log(n)^{1+\epsilon}n^{-1}) \), $\forall\epsilon>0$.
  \item\label{item:2} If $\alpha<1$ then
    \(\frac{1}{K}\sum_{k=1}^{K} \E_{P}{\left[ \Vert
        \hat{\phi}_n(\data_n^{-k}) - F_P \Vert_{P}^2 \right]} = C_P +
    \bigO(n^{-\alpha}) \).
  \end{enumerate}
\end{corollary}


While Proposition~\ref{prop:oracle-prop} provided a precise
finite-sample bound on the average price we pay for using
cross-validation, Corollary~\ref{cor:asymp-cons} states that this
price is asymptotically vanishing, up to poly-logarithmic terms, as
long as the number of learners in the library grows with sample size
at a polynomial rate. The situation \( C_P=0 \) corresponds to a
setting where the library includes a consistent learner.
Cases~\ref{item:1} and~\ref{item:2} correspond to situations where the
oracle learner achieves, respectively, a parametric or non-parametric
asymptotic rate of convergence.

To illustrate the content of Corollary~\ref{cor:asymp-cons}, consider
first a situation where we use a library with an increasing number of
semi-parametric Cox models with different interaction terms, as well
as several Poisson regression models based on different
discretisations of the time scale. Each of these models will
independently achieve a parametric rate of convergence, and hence
item~\ref{item:1} of Corollary~\ref{cor:asymp-cons} states that the
joint survival super learner based on this library will achieve a
near-parametric rate of convergence. The constant \( C_P \) can be
taken equal to the distance to the least false model in the library,
and so the joint survival super learner will approximate the least
false model in the library at a near-parametric rate. Another
situation appears if we add more flexible models to the library, such
as Cox lasso or random survival forests. These models typically
converge at non-parametric rates, with the fastest rate depending on
the unknown data-generating distribution. Item~\ref{item:2} of
Corollary~\ref{cor:asymp-cons} shows that the joint survival super
learner achieves the same convergence rate as the best-performing
algorithm in the library, without any knowledge of the data-generating
distribution.

\section{Numerical experiments}
\label{sec:numer-exper}


In this section we report results from a simulation study where we consider
estimation of the conditional survival function. In the first part, we compare
the joint survival super learner to two IPCW based discrete super learners that use either the
Kaplan-Meier estimator or a Cox model to estimate the censoring probability
\citep{gonzalez2021stacked}. In the second part we compare the joint survival super learner to
the super learner proposed by \cite{westling2021inference}.

In both parts we use the same data-generating mechanism. We generate data
according to a distribution motivated from a real data set in which censoring
depends on the baseline covariates. We simulate data based on the prostate
cancer study of \cite{kattan2000pretreatment}. The outcome of interest is the
time to tumour recurrence, and five baseline covariates are used to predict
outcome: prostate-specific antigen (PSA, ng/mL), Gleason score sum (GSS, values
between 6 and 10), radiation dose (RD), hormone therapy (HT, yes/no) and
clinical stage (CS, six values). The study was designed such that a patient's
radiation dose depended on when the patient entered the study
\citep{gerds2013estimating}. This in turn implies that the time of censoring
depends on the radiation dose. The data were re-analysed in
\citep{gerds2013estimating} where a sensitivity analysis was conducted based on
simulated data. Here we use the same simulation setup, where event and censoring
times are generated according to parametric Cox-Weibull models estimated from
the original data, and the covariates are generated according to either marginal
Gaussian normal or binomial distributions estimated from the original data
\citep[c.f.,][Section~4.6]{gerds2013estimating}. We refer to this simulation
setting as `dependent censoring'. We also considered a simulation setting where
data were generated in the same way, except that censoring was generated
completely independently. We refer to this simulation setting as `independent
censoring'.

For all super learners we use a library consisting of three
learners: The Kaplan-Meier estimator
\citep{kaplan1958nonparametric,Gerds_2019prodlim}, a Cox model
with main effects \citep{cox1972regression, survival-package},
and a random survival forest
\citep{ishwaran2008random,randomForestSRC}. We use the same
library to learn the outcome distribution and the censoring
distribution. The three learners in our library of
learners can be used to learn the cumulative hazard functions
of the outcome and the censoring distribution. The latter
works by training the learner on the data set \( \data_n^c \),
where \( \data_n^c = \{O_i^c\}_{i=1}^n \) with
\( O_i^c = (\tilde{T}_i, 1-\Delta_i, X_i) \). When we say that
we use a learner for the cumulative hazard function of the
outcome to learn the cumulative hazard function of the
censoring time, we mean that the learner is trained on
\( \data_n^c \).

We compare the joint survival super learner to two IPCW based super learners: The
first super learner, called IPCW(Cox), uses a Cox model with main
effects to estimate the censoring probabilities, while the second
super learner, called IPCW(KM), uses the Kaplan-Meier estimator to
estimate the censoring probabilities. The Cox model for the censoring
distribution is correctly specified in both simulation settings while
the Kaplan Meier estimator only estimates the censoring model
correctly in the simulation setting where censoring is
independent. Both IPCW super learners are fitted using the
\texttt{R}-package \texttt{riskRegression}
\citep{Gerds_Ohlendorff_Ozenne_2023}.
%
% | time | sim_setting | true_events | true_cens | at_risk |
% |------+-------------+-------------+-----------+---------|
% |   36 | original    |      24.619 |    61.853 |  25.774 |
% |   36 | indep_cens  |      24.674 |    38.740 |  46.141 |
%
The IPCW super learners use the integrated Brier score up to a fixed time
horizon (36 months). The marginal risk of the event before this time horizon is
\(\approx 24.6\)\%. Under the `dependent censoring' setting the marginal
censoring probability before the time horizon is \(\approx 61.9\)\%. Under the
`independent censoring' setting the marginal censoring probability before this
time horizon is \( \approx 38.7 \)\%.

Each super learner provides a learner for the cumulative
hazard function for the outcome of interest. From the
cumulative hazard function a risk prediction model can be
obtained as described in Section~\ref{sec:use-cases-state}
(with the special case of $\Lambda_2 = 0$). We measure the
performance of each super learner by calculating the index of
prediction accuracy (IPA) \citep{kattan2018index} at a fixed
time horizon (36 months) for the risk prediction model
provided by the super learner. The IPA is 1 minus the ratio
between the model's Brier score and the null model's Brier
score, where the null model is the model that does not use any
covariate information. The IPA is approximated using a large
(\( n = 20,000 \)) independent data set of uncensored data. As
a benchmark we calculate the performance of the risk
prediction model chosen by the oracle selector, which uses the
large data set of uncensored event times to select the learner
with the highest IPA.

The results are shown in Figure~\ref{fig:ipcw-fail}. We see that in
the scenario where censoring depends on the covariates, using the
Kaplan-Meier estimator to estimate the censoring probabilities
provides a risk prediction model with an IPA that is lower than the
risk prediction model provided by the joint survival super learner. The performance
of the risk prediction model selected by the joint survival super learner is similar
to the risk prediction model selected by the IPCW(Cox) super learner
which a priori uses a correctly specified model for the censoring
distribution. Both these risk prediction models are close to the
performance of the oracle, except for small sample sizes.


\begin{figure}
\figuresize{0.7}
\figurebox{20pc}{25pc}{}[experiment-fig-sl-ipcw]
\caption{For the risk prediction models provided by each of the
    super learners, the IPA is plotted against sample size. The
    results are averages across 1000 simulated data sets and the error
    bars are used to quantify the Monte Carlo uncertainty. JSSL
    denotes the joint survival super learner. }
\label{fig:ipcw-fail}
\end{figure}

We next compare the joint survival super learner to the super learner
survSL \citep{westling2021inference}. This is another super learner
which like the joint survival super learner works without a
pre-specified censoring model. Both the joint survival super learner
and survSL provide estimates of the event-time and censoring
distributions. Hence, we compare the performance of these methods with
respect to both the outcome and the censoring distribution. Again we
use the IPA to quantify the predictive performance.

The results are shown in Figures~\ref{fig:zelefski-out}
and~\ref{fig:zelefski-cens}. We see that for most sample sizes, the joint survival super learner selected prediction models for both censoring and outcome which
have similar or higher IPA compared to the prediction models selected
by survSL.
\begin{figure}
\figuresize{0.7}
\figurebox{20pc}{25pc}{}[experiment-fig-sl-survSL-out]
\caption{For the risk prediction models of the outcome provided by each
    of the super learners, the IPA at the fixed time horizon is plotted against
    sample size. The results are averages across 1000 repetitions and the error
    bars are used to quantify the Monte Carlo uncertainty. }
\label{fig:zelefski-out}
\end{figure}

\begin{figure}
\figuresize{.7}
\figurebox{20pc}{25pc}{}[experiment-fig-sl-survSL-cens]
\caption{For the risk prediction models of the censoring model
    provided by each of the super learners, the IPA at the fixed time
    horizon is plotted against sample size. The results are averages
    across 1000 repetitions and the error bars are used to quantify
    the Monte Carlo uncertainty. JSSL denotes the joint survival super
    learner.}
\label{fig:zelefski-cens}
\end{figure}


\section{Prostate cancer study}
\label{sec:real-data-appl}

In this section we use the prostate cancer data of
\cite{kattan2000pretreatment} to illustrate the use of the joint
survival super learner in the presence of competing risks. We have
introduced the data in Section~\ref{sec:numer-exper}. The data
consists of 1,042 patients who are followed from start of followup
until tumour recurrence, death without tumour recurrence or end of
followup (censored) whatever came first. We use the joint survival
super learner to rank libraries of learners for the cause-specific
cumulative hazard functions of tumour recurrence, death without tumour
recurrence, and censoring. The libraries of learners each include five
learners: the Nelson-Aalen estimator, three Cox regression models
(unpenalized, lasso, elastic net) each including additive effects of
the 5 covariates (c.f., Section~\ref{sec:numer-exper}), and a random
survival forest. We use the same set of learners to learn the
cumulative hazard function of tumour recurrence \( \Lambda_1 \), the
cumulative hazard function of death without tumour recurrence
\( \Lambda_2 \), and the cumulative hazard function of the conditional
censoring distribution $\Gamma$.

This gives a library consisting of \( 5^3 = 125 \) learners for the
conditional state occupation probability function \( F \) defined in
equation~(\ref{eq:F-def}). We use five folds for training and testing
the models, and we repeat training and evaluation five times with
different splits.  The integrated Brier score (defined in
Section~\ref{sec:super-learner-simple}) for all learners are shown in
Figure~\ref{fig:zelefski-real}. We see that the prediction
performance is mostly affected by the choice of learner for the
censoring distribution. Several combinations of learners give similar
performance as measured by the integrated Brier score, as long as a
random forest is used to model the censoring distribution.

\begin{figure}
\figuresize{.7}
\figurebox{20pc}{25pc}{}[real-data-state-learner]
\caption{The results of applying the 125 combinations of learners to the
    prostate cancer data set. The error bars are based on five repetitions using
    different splits. We refer to learners of \( \Lambda_1 \), \( \Lambda_2 \),
    and $\Gamma$ as `Tumour learner', `Mortality learner', and `Censoring
    learner', respectively.}
\label{fig:zelefski-real}
\end{figure}


\section{Discussion}
\label{sec:discussion}

A major advantage of the joint survival super learner is that the performance of each
combination of learners can be estimated without additional nuisance
parameters. A potential drawback of our approach is that we are
evaluating the loss of the learners on the level of the observed data
distribution while the target of the analysis is either the event-time
distribution, or the censoring distribution, or both.
Specifically, the finite-sample oracle inequality in
Proposition~\ref{prop:oracle-prop} concerns the function \( F \), which
is a feature of \( P \in \mathcal{P} \), while what we are typically
interested in is \( \Lambda_j \) or \( S \), which are features of
\( Q \in \mathcal{Q} \). We emphasize that while the joint survival super learner
provides us with estimates of \( \Lambda_j \) and $\Gamma$ based on
libraries \( \mathcal{A}_j \) and \( \mathcal{B} \), the performance
of these learners is not assessed directly for their respective target
parameters, but only indirectly via the performance of \( F \).  For
settings without competing risks, our numerical studies suggest that
measuring the performance of \( F \) also leads to good performance
for estimation of \( S \).

Our proposed super learner can be implemented with a broad library of learners
and using existing software.
% For instance the \texttt{R}-package
% \texttt{riskRegression} \citep{Gerds_Ohlendorff_Ozenne_2023} and the \texttt{R}-package by westling
Furthermore, while
the library \( \mathcal{F}(\mathcal{A}_1,\mathcal{A}_2,\mathcal{B}) \) consists
of \( |\mathcal{A}_1||\mathcal{A}_2||\mathcal{B}| \) many learners, we only need to fit
\( |\mathcal{A}_1| +|\mathcal{A}_2| + |\mathcal{B}| \) many learners in each fold. To
evaluate the performance of each learner we need to perform
\( |\mathcal{A}_1||\mathcal{A}_2||\mathcal{B}| \) many operations to calculate the
integrated Brier score in each hold-out sample, one for each combination of the
fitted models, but these operations are often negligible compared to fitting the
models. Hence the joint survival super learner is essentially not more computationally demanding
than any procedure that uses super learning to learn $\Lambda_1$, $\Lambda_2$,
and $\Gamma$ separately. While our proposal is based on constructing the library
\( \mathcal{F} \) from libraries for learning \( \Lambda_1 \), $\Lambda_2$, and
$\Gamma$, it could also be of interest to consider learners that estimate
\( F \) directly.

In our numerical studies, we only considered learners of $\Lambda_j$ and
$\Gamma$ that provide cumulative hazard functions which are piecewise constant
in the time argument. This simplifies the calculation of \( F \) as the
integrals in equation~(\ref{eq:transition}) reduce to sums. When $\Lambda_j$ or
\( \Gamma \) are absolutely continuous in the time argument, calculating \( F \)
is more involved, but we expect that a good approximation can be achieved by
discretisation.


Our original motivation for developing the joint survival super
learner was for use within the framework of targeted or debiased
machine learning -- a general methodology that combines flexible,
data-adaptive estimation with asymptotically valid inference for
low-dimensional target parameters
\citep{van2011targeted,chernozhukov2018double}. In settings with
right-censored competing risks, the relevant nuisance parameters often
include the cause-specific and censoring cumulative hazard functions
\citep[e.g.,][]{van2003unified,rytgaard2022targeted}. The joint
survival super learner immediately provides estimates of these
nuisance parameters and is hence particularly well suited for targeted
and debiased machine learning. We leave the study of the joint
survival super learner in the context of targeted and debiased machine
learning for a future paper.

\appendix

\section*{Appendix}


Define
\( \bar{B}_{\tau,P}(F, o) = \bar{B}_{\tau}(F, o) -
\bar{B}_{\tau}(F_P, o) \) and
\( R_{P}(F) = \E_P{[\bar{B}_{\tau,P}(F, O)]} \), where the
integrated Brier score \( \bar{B}_{\tau} \) was defined in
Section~\ref{sec:super-learner-simple}. Recall the norm
\( \Vert \blank \Vert_{P}\) defined in
equation~(\ref{eq:norm}).

\begin{lemma}
  \label{lemma:norm}
  \( R_{P}(F) = \Vert F - F_P \Vert_{P}^2 \).
\end{lemma}
\begin{proof}[Proof]
  For any \( t \in [0, \tau] \) and \( l\in \{-1,0,1,2\} \) we have
  \begin{align*}
    & \E_{P}{\left[ (F(t, l, X) - \1{\{\eta(t) = l \}})^2 \right]}
    \\
    & =    \E_{P}{\left[ (F(t, l, X) - F_P(t, l, X) + F_P(t, l, X) - \1{\{\eta(t) = l
      \}})^2 \right]}
    \\
    & =    \E_{P}{\left[ (F(t, l, X) - F_P(t, l, X))^2\right]}
      + \E_{P}{\left[ (F_P(t, l, X) - \1{\{\eta(t) = l \}})^2\right]}
    \\
    & \quad
      + 2\E_{P}{\left[ (F(t, l, X) - F_P(t, l, X))(F_P(t, l, X) - \1{\{\eta(t) = l
      \}})\right]}
    \\
    & =    \E_{P}{\left[ (F(t, l, X) - F_P(t, l, X))^2\right]}
      + \E_{P}{\left[ (F_P(t, l, X) - \1{\{\eta(t) = l \}})^2\right]},
  \end{align*}
  where the last equality follows from the tower property. Hence, using Fubini,
  we have
  \begin{equation*}
    \E_P{[\bar{B}_{\tau}(F, O)]}
    = \Vert F - F_P \Vert_{P}^2 + \E_P{[\bar{B}_{\tau}(F_P, O)]}.
  \end{equation*}
\end{proof}

\begin{proof}[Proof of Proposition~\ref{prop:stric-prop}]
  The result follows from Lemma~\ref{lemma:norm}.
\end{proof}

Recall that we use \( \mathcal{F}_n \) to denote a library of learners for the
function \( F \), and that \( \hat{\phi} \) and \( \tilde{\phi} \) denotes,
respectively, the discrete super learner and the oracle learner for the library
\( \mathcal{F}_n \), c.f., Section~\ref{sec:super-learner-simple}.

\begin{proof}[Proof of Proposition~\ref{prop:oracle-prop}]
  Minimizing the loss \( \bar{B}_{\tau} \) is equivalent to
  minimizing the loss \( \bar{B}_{\tau,P} \), so the discrete super learner and
  oracle according to \( \bar{B}_{\tau} \) and \( \bar{B}_{\tau,P} \) are
  identical. By Lemma~\ref{lemma:norm}, \( R_P(F) \geq 0 \) for any
  \( F \in \mathcal{H}_{\mathcal{P}} \), and so using Theorem 2.3 from
  \citep{vaart2006oracle} with \( p=1 \), we have that for all \( \delta >0 \),
\begin{align*}
  & \frac{1}{K} \sum_{k=1}^{K} \E_{P}{\left[ R_P(\hat{\phi}_n(\data_n^{-k})) \right]}
  \\
  &  \quad \leq
    (1+2\delta)\frac{1}{K} \sum_{k=1}^{K}\E_{P}{\left[ R_P(\tilde{\phi}_n(\data_n^{-k})) \right]}
  \\
  & \qquad + (1+\delta) \frac{16 K}{n}
    \log(1 + |\mathcal{F}_n|)\sup_{F \in \mathcal{H}_{\mathcal{P}}}
    \left\{
    M(F) + \frac{v(F)}{R_P(F)}
    \left(
    \frac{1}{\delta} + 1
    \right)
    \right\}
\end{align*}
where for each \( F \in \mathcal{H}_{\mathcal{P}} \),
\( (M(F), v(F)) \) is some Bernstein pair for the function
\(o \mapsto \bar{B}_{\tau,P}(F, o) \). As
\( \bar{B}_{\tau,P}(F, \blank) \) is uniformly bounded by \( \tau \)
for any \( F \in \mathcal{H}_{\mathcal{P}} \), it follows from section
8.1 in \citep{vaart2006oracle} that
\( (\tau, 1.5 \E_P{[\bar{B}_{\tau,P}(F, O)^2]}) \) is a Bernstein
pair for \( \bar{B}_{\tau,P}(F, \blank) \). Now, for any
\( a,b,c \in \R \) we have
\begin{align*}
  (a-c)^2 - (b-c)^2
  & = (a-b+b-c)^2 - (b-c)^2
  \\
  & = (a-b)^2 + (b-c)^2 +2(b-c)(a-b) - (b-c)^2
  \\
  & = (a-b)
    \left\{
    (a-b) +  2(b-c)
    \right\}
  \\
  & = (a-b)
    \left\{
     a + b -2c
    \right\},
\end{align*}
so using this with \( a=F(t, l, x) \), \( b=F_P(t, l, x) \), and
\( c = \1{\{\eta(t) = l\}} \), we have by Jensen's inequality
\begin{align*}
  & \E_P{[\bar{B}_{\tau,P}(F, O)^2]}
  \\
  & \leq
    2\tau\E_{P}{\left[
    \sum_{l=-1}^{2} \int_0^{\tau}
    \left\{
    \left(
    F(t, l, X) - \1{\{\eta(t) = l\}}
    \right)^2
    -
    \left(
    F_P(t, l, X) - \1{\{\eta(t) = l\}}
    \right)^2
    \right\}^2
    \diff t 
    \right]}
  \\
  & =2\tau
    \E_{P}\Bigg[
    \sum_{l=-1}^{2} \int_0^{\tau}
    \left(
    F(t, l, X) - F_P(t, l, X)
    \right)^2
  \\
  & \quad \quad \quad\quad \quad \quad \times
    \left\{
    F(t, l, X) +  F_P(t, l, X)-2 \1{\{\eta(t) = l\}}
    \right\}^2
    \diff t 
    \Bigg]
  \\
  & \leq
    8\tau \E_{P}{\left[
    \sum_{l=-1}^{2} \int_0^{\tau}
    \left(
    F(t, l, X) - F_P(t, l, X)
    \right)^2
    \diff t 
    \right]}.
  \\
  & =
    8\tau \Vert F - F_P \Vert_{P}^2.
\end{align*}
Thus when \( v(F) = 1.5 \E_P{[\bar{B}_{\tau,P}(F, O)^2]} \) we have by
Lemma~\ref{lemma:norm}
\begin{equation*}
  \frac{v(F)}{R_P(F)}
  = 1.5 \frac{\E_P{[\bar{B}_{\tau,P}(F, O)^2]}}{\E_P{[\bar{B}_{\tau,P}(F, O)]}}
  \leq 12 \tau,
\end{equation*}
and so using the Bernstein pairs \( (\tau, 1.5 \E_P{[\bar{B}_{\tau,P}(F, O)^2]}) \) we have
\begin{equation*}
  \sup_{F \in \mathcal{H}_{\mathcal{P}}}
  \left\{
    M(F) + \frac{v(F)}{R_P(F)}
    \left(
      \frac{1}{\delta} + 1
    \right)
  \right\}
  \leq \tau
  \left(
    13 + \frac{12}{\delta}
  \right).
\end{equation*}
For all $\delta>0$ we thus have
\begin{align*}
  \frac{1}{K} \sum_{k=1}^{K} \E_{P}{\left[ R_P(\hat{\phi}_n(\data_n^{-k})) \right]}
  \leq
  &(1+2\delta)\frac{1}{K} \sum_{k=1}^{K}\E_{P}{\left[ R_P(\tilde{\phi}_n(\data_n^{-k})) \right]}
  \\
  & \quad
    + (1+\delta)\log(1 + |\mathcal{F}_n|) \tau \frac{16 K}{n}
    \left(
    13 + \frac{12}{\delta}
    \right),
\end{align*}
and then the final result follows from Lemma~\ref{lemma:norm}.
\end{proof}

\begin{proof}[Proof of Corollary~\ref{cor:asymp-cons}]
  By definition of the oracle and Lemma~\ref{lemma:norm},
  \begin{equation*}
    \frac{1}{K} \sum_{k=1}^{K} \E_{P}{\left[ \Vert \tilde{\phi}_n(\data_n^{-k}) - F_P \Vert_{P}^2
      \right]} \leq
    \frac{1}{K} \sum_{k=1}^{K}\E_{P}{\left[ \Vert
        \phi_n(\data_n^{-k}) - F_P \Vert_{P}^2
      \right]}
    =
    \E_{P}{\left[ \Vert \phi_n(\data_n^{-k}) - F_P \Vert_{P}^2
      \right]},
  \end{equation*}
  for all \( n \in \N \), where the last equality follows because all
  the training sets \( \data_n^{-k} \) have the same distribution. The
  result then follows from Proposition~\ref{prop:oracle-prop} by
  letting $\delta$ grow to zero with \( n \), for instance as
  $\delta_n = \log(n)^{-\epsilon}$ for some $\epsilon>0$.
\end{proof}






\bibliographystyle{biometrika}
\bibliography{bib.bib}

 




\end{document}\vspace{0.4cm} 
