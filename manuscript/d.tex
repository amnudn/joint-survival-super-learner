\documentclass[lineno]{biometrika}
%DIF LATEXDIFF DIFFERENCE FILE
%DIF DEL a.tex   Tue Jul  8 17:40:47 2025
%DIF ADD b.tex   Tue Jul  8 17:41:05 2025

\usepackage{amsmath}
%\usepackage{graphics}

%% Please use the following statements for
%% managing the text and math fonts for your papers:
%\usepackage{times}
%\usepackage[cmbold]{mathtime}
%\usepackage{bm}

\usepackage{newtxtext}
\usepackage[subscriptcorrection]{newtxmath}

% \usepackage{natbib}

\graphicspath{{./art/}}

\usepackage[plain,noend]{algorithm2e}

\makeatletter
\renewcommand{\algocf@captiontext}[2]{#1\algocf@typo. \AlCapFnt{}#2} % text of caption
\renewcommand{\AlTitleFnt}[1]{#1\unskip}% default definition
\def\@algocf@capt@plain{top}
\renewcommand{\algocf@makecaption}[2]{%
  \addtolength{\hsize}{\algomargin}%
  \sbox\@tempboxa{\algocf@captiontext{#1}{#2}}%
  \ifdim\wd\@tempboxa >\hsize%     % if caption is longer than a line
    \hskip .5\algomargin%
    \parbox[t]{\hsize}{\algocf@captiontext{#1}{#2}}% then caption is not centered
  \else%
    \global\@minipagefalse%
    \hbox to\hsize{\box\@tempboxa}% else caption is centered
  \fi%
  \addtolength{\hsize}{-\algomargin}%
}
\makeatother

%%% User-defined macros should be placed here, but keep them to a minimum.
\def\Bka{{\it Biometrika}}

\def\AIC{\textsc{aic}}
\def\T{{ \mathrm{\scriptscriptstyle T} }}
\def\v{{\varepsilon}}

% Notation
\usepackage{dsfont,enumitem, url}
\DeclareMathOperator{\E}{\mathbb{E}} % expectation
\newcommand{\Z}{\mathbb{Z}}
\newcommand{\R}{\mathbb{R}}
\newcommand{\N}{\mathbb{N}}
\newcommand{\C}{\mathbb{C}}
\newcommand{\blank}{\makebox[1ex]{\textbf{$\cdot$}}}
\newcommand\independent{\protect\mathpalette{\protect\independenT}{\perp}}
\def\independenT#1#2{\mathrel{\rlap{$#1#2$}\mkern2mu{#1#2}}}
\renewcommand{\phi}{\varphi}
\renewcommand{\epsilon}{\varepsilon}
\newcommand*\diff{\mathop{}\!\mathrm{d}}
\newcommand{\weakly}{\rightsquigarrow}
\newcommand\smallO{\textit{o}}
\newcommand\bigO{\textit{O}}
\newcommand{\midd}{\; \middle|\;}
\newcommand{\1}{\mathds{1}}
\usepackage{ifthen} %% Empirical process with default argument
\newcommand{\G}[2][n]{
{\ensuremath{\mathbb{G}_{#1}}{\left[#2\right]}}
}
\DeclareMathOperator*{\argmin}{\arg\!\min}
\DeclareMathOperator*{\argmax}{\arg\!\max}
\newcommand{\data}{\ensuremath{\mathcal{D}}}


%\addtolength\topmargin{35pt}
\DeclareMathOperator{\Thetabb}{\mathcal{C}}

% \usepackage[colorlinks,allcolors=blue]{hyperref}
%DIF PREAMBLE EXTENSION ADDED BY LATEXDIFF
%DIF UNDERLINE PREAMBLE %DIF PREAMBLE
\RequirePackage[normalem]{ulem} %DIF PREAMBLE
\RequirePackage{color}\definecolor{RED}{rgb}{1,0,0}\definecolor{BLUE}{rgb}{0,0,1} %DIF PREAMBLE
\providecommand{\DIFadd}[1]{{\protect\color{blue}\uwave{#1}}} %DIF PREAMBLE
\providecommand{\DIFdel}[1]{{\protect\color{red}\sout{#1}}}                      %DIF PREAMBLE
%DIF SAFE PREAMBLE %DIF PREAMBLE
\providecommand{\DIFaddbegin}{} %DIF PREAMBLE
\providecommand{\DIFaddend}{} %DIF PREAMBLE
\providecommand{\DIFdelbegin}{} %DIF PREAMBLE
\providecommand{\DIFdelend}{} %DIF PREAMBLE
\providecommand{\DIFmodbegin}{} %DIF PREAMBLE
\providecommand{\DIFmodend}{} %DIF PREAMBLE
%DIF FLOATSAFE PREAMBLE %DIF PREAMBLE
\providecommand{\DIFaddFL}[1]{\DIFadd{#1}} %DIF PREAMBLE
\providecommand{\DIFdelFL}[1]{\DIFdel{#1}} %DIF PREAMBLE
\providecommand{\DIFaddbeginFL}{} %DIF PREAMBLE
\providecommand{\DIFaddendFL}{} %DIF PREAMBLE
\providecommand{\DIFdelbeginFL}{} %DIF PREAMBLE
\providecommand{\DIFdelendFL}{} %DIF PREAMBLE
%DIF LISTINGS PREAMBLE %DIF PREAMBLE
\RequirePackage{listings} %DIF PREAMBLE
\RequirePackage{color} %DIF PREAMBLE
\lstdefinelanguage{DIFcode}{ %DIF PREAMBLE
%DIF DIFCODE_UNDERLINE %DIF PREAMBLE
  moredelim=[il][\color{red}\sout]{\%DIF\ <\ }, %DIF PREAMBLE
  moredelim=[il][\color{blue}\uwave]{\%DIF\ >\ } %DIF PREAMBLE
} %DIF PREAMBLE
\lstdefinestyle{DIFverbatimstyle}{ %DIF PREAMBLE
	language=DIFcode, %DIF PREAMBLE
	basicstyle=\ttfamily, %DIF PREAMBLE
	columns=fullflexible, %DIF PREAMBLE
	keepspaces=true %DIF PREAMBLE
} %DIF PREAMBLE
\lstnewenvironment{DIFverbatim}{\lstset{style=DIFverbatimstyle}}{} %DIF PREAMBLE
\lstnewenvironment{DIFverbatim*}{\lstset{style=DIFverbatimstyle,showspaces=true}}{} %DIF PREAMBLE
%DIF END PREAMBLE EXTENSION ADDED BY LATEXDIFF

\begin{document}

\jname{Biometrika}
%% The year, volume, and number are determined on publication
\jyear{2025}
\jvol{112}
\jnum{1}
\cyear{2025}
%% The \doi{...} and \accessdate commands are used by the production team
%\doi{10.1093/biomet/asm023}
\accessdate{Advance Access publication on 14 February 2025}

%% These dates are usually set by the production team
\received{2 January 2024}
\revised{3 February 2025}

%% The left and right page headers are defined here:
\markboth{Munch and Gerds}{The joint survival super learner}

%% Here are the title, author names and addresses
\title{The joint survival super learner: A super learner for right-censored
  data}

\author{A. MUNCH} %
\affil{Section of Biostatistics, University of
  Copenhagen \email{a.munch@sund.ku.dk}}

\author{\and T. A. GERDS}
\affil{Section of Biostatistics, University of Copenhagen}

\maketitle

\begin{abstract} Risk prediction models are widely used to guide
real-world decision-making in areas such as healthcare and economics,
and they also play a key role in estimating nuisance parameters in
semiparametric inference. The super learner is a machine learning
framework that combines a library of prediction algorithms into a
meta-learner using cross-validated loss. In the context of
right-censored data, careful consideration must be given to both the
choice of loss function and the estimation of expected loss.
Moreover, estimators such as inverse probability of censoring
weighting (IPCW) require accurate modeling and \DIFdelbegin \DIFdel{estimates }\DIFdelend \DIFaddbegin \DIFadd{an estimator }\DIFaddend of the
censoring distribution. We propose a novel approach to super learning
for survival analysis that jointly evaluates candidate learners for
both the event-time distribution and the censoring distribution. Our
method imposes no restrictions on the algorithms included in the
library, accommodates competing risks, and does not rely on a single
pre-specified estimator of the censoring distribution. We establish \DIFdelbegin \DIFdel{theoretical guarantees for our proposed
  method, including }\DIFdelend a
finite-sample \DIFdelbegin \DIFdel{oracle inequality. In a simulation
  study, our super learner was able to better account for different
  censoring mechanisms than existing methods}\DIFdelend \DIFaddbegin \DIFadd{bound on the average price we pay for using
cross-validation, and show that this price vanishes asymptotically, up
to poly-logarithmic terms, provided that the size of the library does
not grow faster than at a polynomial rate in the sample size}\DIFaddend . We
demonstrate the practical utility of our method using prostate cancer
data \DIFaddbegin \DIFadd{and compare it to existing super learner algorithms for survival
analysis using synthesized data}\DIFaddend .
\end{abstract}

\begin{keywords}
competing risks; cross-validation; loss based estimation; right-censored data; super learner
\end{keywords}

\section{Introduction}
\label{sec:introduction}

Accurately predicting risk from time-to-event data is a central
challenge in \DIFdelbegin \DIFdel{diverse }\DIFdelend \DIFaddbegin \DIFadd{various }\DIFaddend research fields, \DIFdelbegin \DIFdel{including }\DIFdelend \DIFaddbegin \DIFadd{such as }\DIFaddend epidemiology, economics,
and weather forecasting, with applications in clinical decision making
and policy interventions. For instance, in prostate cancer management,
clinicians often need to estimate a patient’s risk of disease
progression and mortality over time to make informed decisions about
treatment strategies such as active surveillance versus immediate
intervention. Reliable \DIFdelbegin \DIFdel{time-to-event risk predictions
}\DIFdelend \DIFaddbegin \DIFadd{risk prediction models }\DIFaddend can help tailor care to
individual patients, avoid overtreatment, and allocate healthcare
resources more effectively. Super learning \citep{van2007super}, also
known as ensemble learning or stacked regression
\citep{wolpert1992stacked,breiman1996stacked}, provides a powerful
approach to this problem by combining multiple candidate prediction
models to reduce the risk of bias incurred by a single potentially
mispecified model. In survival analysis, a super learner may \DIFaddbegin \DIFadd{for example }\DIFaddend combine a
stack of Cox regression models with a stack of random survival forests
\citep[][Section 8.4]{gerds2021medical}. Such a strategy has recently
produced KDpredict (\url{https://kdpredict.com/}) a model which
jointly predicts the risks of kidney failure and all-cause mortality
at multiple time horizons based on different sets of covariates
\citep{liu2024predicting}. To evaluate the prediction performance of
the learners, the super learner behind KDpredict uses inverse
probability of censoring weighting (IPCW), where the censoring
distribution is estimated under the restrictive assumption that it
does not depend on the covariates. This is a potential source of bias
which is difficult to overcome with the currently available methods.

In this \DIFdelbegin \DIFdel{paper}\DIFdelend \DIFaddbegin \DIFadd{article}\DIFaddend , we propose the \DIFaddbegin {\it \DIFaddend joint survival super learner\DIFaddbegin }\DIFaddend , a
new super learner designed to handle the specific challenges of
ensemble learning with right-censored data. The joint survival super
learner simultaneously learns prediction models for the event-time and
censoring distributions. The joint survival super learner is based on
\DIFdelbegin \DIFdel{the simple idea of an artificial }\DIFdelend \DIFaddbegin \DIFadd{a }\DIFaddend competing risks model \DIFaddbegin \DIFadd{for the observed data}\DIFaddend , in which censoring is
included as a state of its own\DIFdelbegin \DIFdel{. Candidate learners }\DIFdelend \DIFaddbegin \DIFadd{, such that at any time it is known in
which state an individual is. We assume conditionally independent
censoring and exploit well-known relationships between the observed
data distribution on the one side and the partly unobserved distributions of the event
time and the censoring time on the other. Learners }\DIFaddend for the event-time
and censoring hazard functions are then assessed based on how well
they predict the state occupation probabilities of the \DIFdelbegin \DIFdel{artificial }\DIFdelend competing risks
model over time, given baseline covariate information. Our estimation
framework \DIFdelbegin \DIFdel{allows for }\DIFdelend \DIFaddbegin \DIFadd{thus naturally incorporates }\DIFaddend competing risks, avoids
restrictive assumptions on the censoring distribution, and \DIFaddbegin \DIFadd{produces an
estimator for the censoring distribution. Our approach }\DIFaddend is also fully
flexible with respect to the choice of learners. The latter is in
contrast to other proposals which restrict the library of learners to
specific model classes \citep{polley2011-sl-cens,golmakani2020super},
\DIFdelbegin \DIFdel{see
}\DIFdelend \DIFaddbegin \DIFadd{as we discuss in more detail in }\DIFaddend Section~\ref{sec:super-learning}.

To analyse the theoretical properties of the joint survival super
learner, we focus on the discrete super learner, which selects the
model in the library with the best estimated performance
\citep{van2007super}. We provide theoretical guarantees for the
performance of the joint survival super learner, and in particular
show that the discrete joint survival super learner is consistent\DIFdelbegin \DIFdel{under natural conditions and satisfies }\DIFdelend \DIFaddbegin \DIFadd{,
when the library of learners includes at least one consistent
learner. We also derive }\DIFaddend a finite-sample oracle inequality \DIFaddbegin \DIFadd{for the
discrete joint survival super learner}\DIFaddend . We demonstrate how to construct
a library \DIFdelbegin \DIFdel{using common
survival models and
how to obtain risk predictions from the resulting
ensemble}\DIFdelend \DIFaddbegin \DIFadd{of learners using common methods for survival analysis and
illustrate the use of the joint survival super learner using a
prostate cancer data set}\DIFaddend .

The \DIFdelbegin \DIFdel{rest of the paper }\DIFdelend \DIFaddbegin \DIFadd{article }\DIFaddend is organized as follows. We introduce our notation and
framework in Section~\ref{sec:framework}.
Section~\ref{sec:super-learning} introduces loss-based super learning
and \DIFdelbegin \DIFdel{presents }\DIFdelend \DIFaddbegin \DIFadd{discusses other }\DIFaddend existing super learners for right-censored
data. In Section~\DIFdelbegin \DIFdel{\ref{sec:super-learner-simple} }\DIFdelend \DIFaddbegin \DIFadd{\ref{sec:joint-survival-super-learner} }\DIFaddend we define the joint
survival super learner, while Section~\ref{sec:theor-results-prop}
provides theoretical guarantees. Section~\ref{sec:numer-exper} reports
the results of \DIFaddbegin \DIFadd{a series of }\DIFaddend numerical experiments, and
Section~\ref{sec:real-data-appl} illustrates the method on prostate
cancer data. We conclude with a discussion in
Section~\ref{sec:discussion}. Proofs are collected in the
Appendix. Code and an implementation of the joint survival super
learner \DIFaddbegin \DIFadd{in R \mbox{%DIFAUXCMD
\citep{R} }\hspace{0pt}%DIFAUXCMD
}\DIFaddend are available at
\url{https://github.com/amnudn/joint-survival-super-learner}.

\section{Notation and framework}
\label{sec:framework}

In a competing risks framework \citep{andersen2012statistical} \DIFaddbegin \DIFadd{with
\(J\) competing risks}\DIFaddend , let \( T\) be a time to event variable,
\DIFdelbegin \DIFdel{\(D\in\{1,2\}\) }\DIFdelend \DIFaddbegin \DIFadd{\(D\in\{1,2,\dots,J\}\) }\DIFaddend the cause of the event, and $X \in
\mathcal{X}$ a vector of baseline covariates taking values in a
bounded subset \( \mathcal{X} \subset \R^p \), \( p\in\N \). Let
$\tau< \infty$ be a fixed prediction horizon. We use \DIFdelbegin \DIFdel{\(
\mathcal{Q} \) }\DIFdelend \DIFaddbegin \DIFadd{\(\mathcal{Q} \)
}\DIFaddend to denote the collection of all probability measures on \DIFdelbegin \DIFdel{\( [0,\tau] \times \{1,2\}\times \mathcal{X} \) }\DIFdelend \DIFaddbegin \DIFadd{\( [0,\tau]
\times \{1,2,\dots,J\}\times \mathcal{X} \) }\DIFaddend such that \( (T, D, X) \sim Q \)
for some unknown \( Q \in \mathcal{Q} \). For \DIFdelbegin \DIFdel{\(j\in\{1,2\}\)}\DIFdelend \DIFaddbegin \DIFadd{\(j\in\{1,2,
\dots,J\}\)}\DIFaddend , the cause-specific conditional cumulative hazard
functions \( \Lambda_{j} \colon [0, \tau] \times \mathcal{X}
\rightarrow \R_+ \) are defined as
\begin{equation*}
  % \label{eq:cum-haz}
  \Lambda_{j}(t \mid x) = \int_0^t\frac{  Q(T \in \diff s, D=j \mid X=x )}{Q(T \geq s \mid X=x )}.
\end{equation*} For ease of presentation we assume \DIFdelbegin \DIFdel{throughout }\DIFdelend \DIFaddbegin \DIFadd{from now on that
\(J=2\) and }\DIFaddend that the map \( t\mapsto \Lambda_j(t \mid x) \) is
continuous for all \( x \) and \( j \), however, all technical
arguments extend naturally to the general case
\citep{andersen2012statistical}.  The event-free survival function
conditional on covariates is \DIFaddbegin \DIFadd{given by
}\DIFaddend \begin{equation}
  \label{eq:surv-def}
  S(t \mid x)=\exp\left\{-\Lambda_{1}(t \mid x)-\Lambda_{2}(t \mid x)\right\}.
\end{equation}
Let \( \mathcal{M}_{\tau}\) denote the space of all conditional
cumulative hazard functions on \( [0,\tau] \times\mathcal{X}\). Any
distribution \( Q \in \mathcal{Q} \) can be characterized by
\begin{equation*}
  \label{eq:parametrizeQ}
  \begin{split}
    Q(\diff t,j,\diff x)=& \left\{S(t- \mid x)\Lambda_1(\diff t \mid x)H(\diff x)\right\}^{\1{\{j=1\}}}\\
                         &  \left\{S(t- \mid x)\Lambda_2(\diff t \mid x)H(\diff x)\right\}^{\1{\{j=2\}}},
  \end{split}
\end{equation*}
where \(\Lambda_{j} \in \mathcal{M}_{\tau}\) for \(j=1,2\) and \(H\) is the marginal
distribution of the covariates.

We consider the right-censored setting in which we observe \(O =
(\tilde{T},\tilde D, X)\), where $\tilde T = \min(T,C)$ for a
right-censoring time \(C\), $\Delta = \1{\{T \leq C\}}$, and \(\tilde
D=\Delta D\). Let \(\mathcal{P}\) denote a set of probability measures
on the sample space \(\mathcal{O} = [0, \tau] \times \{0, 1, 2\}
\times \mathcal{X}\) such that \(O \sim P \) for some unknown \(P\in
\mathcal{P}\). We assume that the event times and the censoring times
are conditionally independent given covariates, \( T \independent C
\mid X \). This implies that any distribution \( P \in \mathcal{P} \)
is characterized by a distribution \( Q \in \mathcal{Q} \) and a
conditional cumulative hazard function for \( C \) given \( X \)
\citep[c.f.,][]{begun1983information,gill1997coarsening}. We use
\(\Gamma\in\mathcal{M}_{\tau}\) to denote the cumulative hazard
function of the conditional censoring distribution given
covariates. For ease of presentation we assume that \(t\mapsto
\Gamma(t \mid x) \) is continuous for all \( x \). We let
\((t,x)\mapsto G(t \mid x)=\exp\left\{-\Gamma(t \mid x)\right\}\)
denote the survival function of the conditional censoring
distribution. The distribution \( P \) is characterized by
\begin{equation}\label{eq:parametrizeP}
  \begin{split}
    P(\diff t, j, \diff x) =& \left\{G(t- \mid x)S(t- \mid x)\Lambda_1(\diff t \mid x)H(\diff x)\right\}^{\1{{\{j=1\}}}}\\
                            & \left\{G(t- \mid x)S(t- \mid x)\Lambda_2(\diff t \mid x)H(\diff x)\right\}^{\1{{\{j=2\}}}}\\
                            & \left\{G(t- \mid x)S(t- \mid x)\Gamma(\diff t \mid x)H(\diff x)\right\}^{\1{{\{j=0\}}}}\\
    = & \left\{G(t- \mid x)Q(\diff t,j,\diff x)\right\}^{\1{{\{j\ne 0\}}}}\\    
                            & \left\{G(t- \mid x)S(t- \mid x)\Gamma(\diff t \mid x)H(\diff x)\right\}^{\1{{\{j=0\}}}}.
  \end{split}
\end{equation}
Hence, we may write
\( \mathcal{P} = \{ P_{Q, \Gamma} : Q \in \mathcal{Q}, \Gamma \in
\mathcal{G} \} \) for some \( \mathcal{G} \subset \mathcal{M}_{\tau} \). We
also have \(H\)-almost everywhere
\begin{equation*}
P(\tilde T>t \mid X=x) = S(t \mid x)G(t \mid x) = \exp\left\{-\Lambda_{1}(t \mid x)-\Lambda_{2}(t \mid x)-\Gamma(t \mid x) \right\}.
\end{equation*} We assume that there exists \(\kappa<\infty\) such
that \(\Lambda_{j}(\tau- \mid x)<\kappa \), for \(j\in\{1,2\}\), and
\(\Gamma(\tau- \mid x)<\kappa\) for almost all \(x\in\mathcal
X\). This implies that \(G(\tau- \mid x)\) is bounded away
from zero for almost all \(x\in\mathcal X\).  Under these assumptions,
the conditional cumulative hazard functions \(\Lambda_{j}\) and
\(\Gamma\) can be identified from \(P\) by
\begin{align}
  \Lambda_{j}(t \mid x) &= \int_0^t\frac{  P(\tilde T \in \diff s, \tilde D=j \mid X=x )}{P(\tilde T \geq s \mid X=x )}, \label{eq:lambdaj}\\
  \Gamma(t \mid x) &= \int_0^t\frac{  P(\tilde T \in \diff s, \tilde D=0 \mid X=x )}{P(\tilde T \geq s \mid X=x )}\label{eq:gamma}.
\end{align}
Thus, we can consider $\Lambda_j$ and \(\Gamma\) as operators which map from
\( \mathcal{P} \) to \(\mathcal M_{\tau}\).

\section{Loss-based super learning}
\label{sec:super-learning}

Loss-based super learning requires a library of \DIFdelbegin \DIFdel{candidate models (or
}\textit{\DIFdel{learners}}%DIFAUXCMD
\DIFdel{)}\DIFdelend \DIFaddbegin \DIFadd{learners}\DIFaddend , a
cross-validation algorithm, and a loss function for evaluating
predictive performance on hold-out samples. Let \(
\data_n=\{O_i\}_{i=1}^n \in \mathcal{O}^n \) be a data set of i.i.d.\
observations from \( P \in \mathcal{P} \), and $\mathcal{A}$ a
collection of candidate learners. Let \(\Theta\) be the parameter
space, which in our case is a class of functions representing
different models. Each learner \(a \in \mathcal{A}\) is a map \( a
\colon \mathcal{O}^n \rightarrow \Theta \) which takes a data set as
input and returns an estimate $a(\data_n) \in \Theta$. Let \(L\colon
\Theta \times \mathcal{O} \rightarrow \R_+\) be a loss function,
representing the performance of the model $\theta \in \Theta$ at the
observation \( O \in \mathcal{O} \), where lower values mean better
performance.

The expected loss of a learner is estimated by splitting the data set
$\data_n$ into $K$ disjoint approximately equally sized subsets
\(\data_n^1, \data_n^2, \dots, \data_n^K \) and then calculating the
cross-validated loss
\begin{equation}
  \label{eq:cv-risk-est}
  \hat{R}_n(a; L) =
  \frac{1}{K}\sum_{k=1}^{K}
  \frac{1}{| \data_n^{k} |}\sum_{O_i \in \data_n^{k}}
  L
  {
    \left(
      a{ (\data_n^{-k})}
      , O_i
    \right)
  },
  \quad \text{with} \quad
  \data_n^{-k} = \data_n \setminus \data_n^{k}.
\end{equation}
The subset \(\data_n^{-k}\) is referred to as the \(k\)'th training
sample, while \(\data_n^{k}\) is referred to as the \(k\)'th test or
hold-out sample.
The discrete super learner is defined as
\begin{equation*}
\hat{a}_n = \argmin_{a\in\mathcal A}\hat{R}_n(a; L),
\end{equation*}
and depends on both the library of learners and the specific
partitioning of the data into cross-validation folds
\( \data_n^1, \dots, \data_n^K \).

When designing a super learner for right-censored data, particular
care must be taken in the choice of loss function and in the
estimation of the expected loss. A commonly used loss function for
right-censored data is the partial log-likelihood loss
\citep[e.g.,][]{li2016regularized,yao2017deep,lee2018deephit,katzman2018deepsurv,gensheimer2019scalable,lee2021boosted,kvamme2021continuous}.
This loss function is also recommended for super learning with
right-censored data by \cite{polley2011-sl-cens}, under the assumption
that data are observed in discrete time. However, the partial
log-likelihood loss does not work well as a \DIFaddbegin \DIFadd{general purpose }\DIFaddend measure of
performance in hold-out samples \DIFaddbegin \DIFadd{when data are }\DIFaddend observed in continuous
time. The \DIFdelbegin \DIFdel{is because it
}\DIFdelend \DIFaddbegin \DIFadd{reason is that the partial log-likelihood }\DIFaddend assigns an
infinite value to any learner that predicts piecewise constant
cumulative hazard functions, if the test set contains event times that
\DIFdelbegin \DIFdel{were }\DIFdelend \DIFaddbegin \DIFadd{are }\DIFaddend not observed in the training set. This problem occurs with
prominent survival learners including the Kaplan-Meier estimator,
random survival forests, and semi-parametric Cox regression models,
and these learners cannot be included in the library of the super
learner proposed by \cite{polley2011-sl-cens}. When a proportional
hazards model is assumed, the baseline hazard function can be profiled
out of the likelihood \citep{cox1972regression}. The cross-validated
partial log-likelihood loss \citep{verweij1993cross} has therefore
been suggested as a loss function for super learning by
\cite{golmakani2020super}. \DIFdelbegin \DIFdel{This }\DIFdelend \DIFaddbegin \DIFadd{However, this }\DIFaddend choice of loss function
restricts the library of learners to include only Cox proportional
hazards models, and hence excludes many learners such as, e.g., random
survival forests, additive hazards models, and accelerated failure
time models.

Alternative approaches for super learning with right-censored data use
an inverse probability of censoring weighted (IPCW) loss function
\citep{graf1999assessment,van2003unicv,molinaro2004tree,keles2004asymptotically,hothorn2006survival,gerds2006consistent,gonzalez2021stacked},
censoring unbiased transformations
\citep{fan1996local,steingrimsson2019censoring}, or pseudo-values
\citep{andersen2003generalised,mogensen2013random,sachs2019ensemble}.
All these methods rely on an estimator of the censoring distribution,
and their drawback is that this estimator has to be pre-specified.
Recent work by \cite{han2021inverse} and \cite{westling2021inference}
circumvents the need to pre-specify a censoring model by iterating
between estimation of the outcome and censoring models. However, this
iterative procedure is in general not guaranteed to converge to the
true data-generating mechanism
\citep[][Appendix~A.4]{munch2024thesis}. 


\section{The joint survival super learner}
\DIFdelbegin %DIFDELCMD < \label{sec:super-learner-simple}
%DIFDELCMD < 

%DIFDELCMD < %%%
\subsection{\DIFdel{An artificial competing risks model}}
%DIFAUXCMD
\addtocounter{subsection}{-1}%DIFAUXCMD
%DIFDELCMD < \label{sec:an-artif-comp}
%DIFDELCMD < %%%
\DIFdelend \DIFaddbegin \label{sec:joint-survival-super-learner}
\DIFaddend 

The main idea of the joint survival super learner is to \DIFdelbegin \DIFdel{jointly use
learners for }\DIFdelend \DIFaddbegin \DIFadd{specify
libraries of learners for the hazard functions }\DIFaddend \( \Lambda_1 \), \(
\Lambda_2 \), and \( \Gamma \), and \DIFaddbegin \DIFadd{to exploit }\DIFaddend the relations in
equation~(\ref{eq:parametrizeP}) \DIFdelbegin \DIFdel{, to learn a feature
of the observed data distribution \( P \). The
discrete }\DIFdelend \DIFaddbegin \DIFadd{to define a joint loss function. The
}\DIFaddend joint survival super learner \DIFdelbegin \DIFdel{ranks }\DIFdelend \DIFaddbegin \DIFadd{thus evaluates }\DIFaddend a tuple of learners for \(
(\Lambda_1, \Lambda_2, \Gamma) \) based on how well they jointly
\DIFdelbegin \DIFdel{model }\DIFdelend \DIFaddbegin \DIFadd{predict }\DIFaddend the observed data \DIFaddbegin \DIFadd{and the discrete joint survival super
learner chooses the best performing tupel}\DIFaddend . To formally introduce the
joint survival super learner, we define the process
\begin{equation*}
  \eta(t) = \1\{\tilde{T} \leq t, \tilde D=1\} + 2\,\1\{\tilde{T} \leq t, \tilde
  D=2\} - \1\{\tilde{T} \leq t, \tilde D=0\},
  \quad \text{for} \quad t \in [0, \tau],
\end{equation*}
which takes values in \( \{-1,0,1,2\}\). The four values
represent four mutually exclusive states. Specifically, value
\( 0 \) represents the state where the individual is still
event-free and uncensored, value \( 1\) the state where the
event of interest has occurred, value \( 2\) the state where a
competing risk has occurred, and value \( -1\) the state where
the observation is right-censored. The state occupation
probabilities given baseline covariates \( X \) are given by
the function
\begin{equation}
  \label{eq:F-def}
  F(t, l, x) = P(\eta(t) = l \mid X=x),
\end{equation}
for all \( t \in [0,\tau] \), \( l \in \{-1,0,1,2\} \), and
\( x \in \mathcal{X} \).

The joint survival super learner is a super learner for the
function-valued parameter $\theta(P) = F$ which is identified through
equation~(\ref{eq:F-def}). Under conditional independent censoring,
each tuple $(\Lambda_{1}, \Lambda_{2}, \Gamma, H)$ characterizes a
distribution \(P\in\mathcal P\), c.f.\
equation~\eqref{eq:parametrizeP}, which in turn determines
\( (F, H) \). Hence, a learner for \( F \) can be constructed from
learners for \( \Lambda_1 \), \( \Lambda_2 \), and $\Gamma$ as
follows:
\begin{equation}\label{eq:transition}
  \begin{split}
    F(t, 0, x)
    &
      = P(\tilde{T} > t \mid X= x)
      =
      \exp{{\{-\Lambda_{1}(t \mid x)-\Lambda_{2}(t \mid x) - \Gamma(t \mid x)\}
      }},
    \\
    F(t, 1, x)
    &
      = P(\tilde{T} \leq t, \tilde{D}=1 \mid X=x)
      = \int_0^t F(s-, 0, x)  \Lambda_{1}(\diff s \mid x),
    \\
    F(t, 2, x)
    &
      = P(\tilde{T} \leq t, \tilde{D}=2 \mid X=x)
      = \int_0^t  F(s-, 0, x)  \Lambda_{2}(\diff s \mid x),
    \\
    F(t, -1, x)
    &
      = P(\tilde{T} \leq t, \tilde{D}=0 \mid X=x)
      = \int_0^t F(s-, 0, x)  \Gamma(\diff s \mid x).
  \end{split}
\end{equation}
Equation~\eqref{eq:transition} implies that a library for the joint survival super learner can by build from three libraries of learners:
\(\mathcal{A}_1\), \( \mathcal{A}_2 \), and \( \mathcal{B} \),
where \(\mathcal{A}_1\) and \( \mathcal{A}_2\) contain
learners for the conditional cause-specific cumulative hazard
functions \(\Lambda_1\) and \( \Lambda_2\), respectively, and
\(\mathcal{B}\) contains learners for the conditional
cumulative hazard function of the censoring distribution.
Taking the Cartesian product of these libraries, we obtain a library
$\mathcal{F}$ of learners for \( F \):
\begin{equation}
  \label{eq:jssl-lib-def}
  \mathcal{F}(\mathcal{A}_1, \mathcal{A}_2, \mathcal{B})
  = \{ \phi_{a_1,a_2, b} : a_1 \in \mathcal{A}_1, a_2 \in \mathcal{A}_2, b \in \mathcal{B}\},
\end{equation}
where in correspondence with the relations in equation
\eqref{eq:transition},
\DIFdelbegin \begin{align*}
    \DIFdel{\phi_{a_1,a_2, b}(\data_n)(t,0,x)
  }&\DIFdel{= \exp{\{-a_1(\data_n)(s \mid x)-a_2(\data_n)(s \mid x) - b(\data_n)(s \mid
    x)\} },
  }\\
  \DIFdel{\phi_{a_1,a_2, b}(\data_n)(t,1,x)
  }&\DIFdel{= \int_0^t
    \phi_{a_1,a_2, b}(\data_n)(s-,0,x)  a_1(\data_n)(\diff s \mid x),
  }\\
  \DIFdel{\phi_{a_1,a_2, b}(\data_n)(t,2,x)
  }&\DIFdel{= \int_0^t\phi_{a_1,a_2, b}(\data_n)(s-,0,x)  a_2(\data_n)(\diff s \mid x),
  }\\
  \DIFdel{\phi_{a_1,a_2, b}(\data_n)(t,-1,x)
  }&\DIFdel{= \int_0^t \phi_{a_1,a_2, b}(\data_n)(s-,0,x)  b(\data_n)(\diff s \mid x).
}\end{align*}%DIFAUXCMD
\DIFdelend \DIFaddbegin \begin{equation}
  \DIFadd{\begin{split}\label{eq:anti-transition}
    \phi_{a_1,a_2, b}(\data_n)(t,0,x)
  &= \exp{\{-a_1(\data_n)(s \mid x)-a_2(\data_n)(s \mid x) - b(\data_n)(s \mid
    x)\} },
  \\
  \phi_{a_1,a_2, b}(\data_n)(t,1,x)
  &= \int_0^t
    \phi_{a_1,a_2, b}(\data_n)(s-,0,x)  a_1(\data_n)(\diff s \mid x),
  \\
  \phi_{a_1,a_2, b}(\data_n)(t,2,x)
  &= \int_0^t\phi_{a_1,a_2, b}(\data_n)(s-,0,x)  a_2(\data_n)(\diff s \mid x),
  \\
  \phi_{a_1,a_2, b}(\data_n)(t,-1,x)
  &= \int_0^t \phi_{a_1,a_2, b}(\data_n)(s-,0,x)  b(\data_n)(\diff s \mid x).
  \end{split}
}\end{equation}\DIFaddend 
  Notably, the libraries \( \mathcal{A}_1 \), \(
\mathcal{A}_2 \), and \( \mathcal{B} \) can be constructed using
standard software for survival analysis.  \DIFdelbegin \DIFdel{In
}\DIFdelend \DIFaddbegin \DIFadd{For example, in the
}\DIFaddend \texttt{R} \DIFdelbegin \DIFdel{, for instance, we can construct Cox models as
learners }\DIFdelend \DIFaddbegin \DIFadd{software we can specify various ways to include covariates
in a Cox regression model and fit learners of the hazard functions
}\DIFaddend using the \texttt{survival}-package \citep{survival-package}, and we
can \DIFdelbegin \DIFdel{construct random survival forests as learners }\DIFdelend \DIFaddbegin \DIFadd{specify hyper parameters of a random survival forest and derive
learners of the hazard functions }\DIFaddend using the
\texttt{randomForestSRC}-package \citep{randomForestSRC}.

To evaluate how well a function \( F \) predicts the
process $\eta$ we use the integrated Brier score \citep{graf1999assessment}
\DIFdelbegin \DIFdel{\( \bar B_\tau( F,O) = \int_0^{\tau} B_t(F,O) \diff t \),
where \( B_t \) is the Brier score
\mbox{%DIFAUXCMD
\citep{brier1950verification} }\hspace{0pt}%DIFAUXCMD
at time \( t \in [0, \tau] \),
}\begin{displaymath}
  \DIFdel{B_t(F,O) = \sum_{l=-1}^{2}
  \left(
      F(t,l,X) - \1{\{\eta(t)=l\}}
  \right)^2.
}\end{displaymath}%DIFAUXCMD
\DIFdelend \DIFaddbegin \DIFadd{evaulated at time \(\tau\):
}\begin{equation*}
  \DIFadd{\bar B_\tau(F,O) = \int_0^{\tau} \sum_{l=-1}^{2}
  \left(
      F(t,l,X) - \1{\{\eta(t)=l\}}
  \right)^2\diff t.
}\end{equation*}\DIFaddend  \DIFdelbegin \DIFdel{The Brier score is here the squared prediction error across
all of }\DIFdelend \DIFaddbegin \DIFadd{Here the integrand is the average Brier score at time \(t\) across
}\DIFaddend the four states \DIFaddbegin \DIFadd{\mbox{%DIFAUXCMD
\citep{brier1950verification}}\hspace{0pt}%DIFAUXCMD
}\DIFaddend . Based on a split of a
data set \(\data_n\) into $K$ disjoint approximately equally sized
subsets (c.f., Section \ref{sec:super-learning}), each learner \(
\phi_{a_1, a_2, b} \) in the library \( \mathcal{F}(\mathcal{A}_1,
\mathcal{A}_2, \mathcal{B}) \) is evaluated using the cross-validated
loss,
\begin{equation*}
  \hat{R}_{n}(\phi_{a_1,a_2,b} ; \bar{B}_{\tau}) =
  \frac{1}{K}\sum_{k=1}^{K}
  \frac{1}{| \data_n^{k} |}\sum_{O_i \in \data_n^{k}}
  \bar B_\tau
  {
    \left(
      \phi_{a_1,a_2,b}{ (\data_n^{-k})}
      , O_i
    \right)
  },
\end{equation*}
and the discrete joint survival super learner is \DIFdelbegin \begin{align*}\DIFdel{%DIFDELCMD < \label{eq:discrete-state-learner}%%%
  \hat{\phi}_n
  }&\DIFdel{=  \argmin_{(a_1,a_2,b)\in \mathcal{A}_1\times\mathcal{A}_2\times\mathcal{B}}
    \hat{R}_{n}(\phi_{a_1,a_2,b} ; \bar{B}_{\tau}).
}\end{align*}%DIFAUXCMD
\DIFdelend \DIFaddbegin \DIFadd{best performing tupel of hazard functions:
}\begin{equation}\DIFadd{\label{eq:discrete-JSLL}
  (\hat \Lambda_{1n},\hat \Lambda_{2n}, \hat \Gamma_{n})
  =  \argmin_{(a_1,a_2,b)\in \mathcal{A}_1\times\mathcal{A}_2\times\mathcal{B}}
    \hat{R}_{n}(\phi_{a_1,a_2,b} ; \bar{B}_{\tau}).
}\end{equation}\DIFaddend 


\DIFdelbegin \subsection{\DIFdel{Obtaining risk predictions}}
%DIFAUXCMD
\addtocounter{subsection}{-1}%DIFAUXCMD
%DIFDELCMD < \label{sec:use-cases-state}
%DIFDELCMD < 

%DIFDELCMD < %%%
\DIFdel{The joint survival super learner estimates the function \( F \) which
depends on the censoring distribution and is therefore typically not
of direct interest in itself. For instance, in the prostate cancer
example we consider in Sections~\ref{sec:numer-exper}
and~\ref{sec:real-data-appl}, the value \( F(t, l, x) \) denotes the
conditional probability that a patient with baseline characteristics
\( X=x \) will, before time point \( t \):
have had tumour recurrence
before leaving the study ($l=1$); have died without tumour recurrence
before leaving the study ($l=2$); have left the study (state $l=-1$);
or be alive and part of the study without tumour recurrence ($l=0$).
The reference to still being part of the study is irrelevant from a
new patient's perspective. We here demonstrate how clinically relevant
risk predictions can instead be obtained from the joint survival super
learner.
}\DIFdelend %DIF >  The joint survival super learner estimates the function \( F \) which
%DIF >  depends on the censoring distribution and is therefore typically not
%DIF >  of direct interest in itself. For instance, in the prostate cancer
%DIF >  example we consider in Sections~\ref{sec:numer-exper}
%DIF >  and~\ref{sec:real-data-appl}, the value \( F(t, l, x) \) denotes the
%DIF >  conditional probability that a patient with baseline characteristics
%DIF >  \( X=x \) will, before time point \( t \): have had tumour recurrence
%DIF >  before leaving the study ($l=1$); have died without tumour recurrence
%DIF >  before leaving the study ($l=2$); have left the study (state $l=-1$);
%DIF >  or be alive and part of the study without tumour recurrence ($l=0$).
%DIF >  The reference to still being part of the study is irrelevant from a
%DIF >  new patient's perspective. We here demonstrate how clinically relevant
%DIF >  risk predictions can instead be obtained from the joint survival super
%DIF >  learner.

\DIFdelbegin \DIFdel{We recall that we work under }\DIFdelend \DIFaddbegin \DIFadd{Under }\DIFaddend the assumption of conditional independent censoring and
positivity, as \DIFdelbegin \DIFdel{introduced }\DIFdelend \DIFaddbegin \DIFadd{stated }\DIFaddend in Section~\ref{sec:framework}\DIFdelbegin \DIFdel{. Under these assumptions}\DIFdelend , it follows \DIFdelbegin \DIFdel{by
}\DIFdelend \DIFaddbegin \DIFadd{from
}\DIFaddend equations~(\ref{eq:lambdaj}) and~(\ref{eq:gamma}) and the definition
of \( F \) that
\begin{equation}
  \label{eq:7}
  \Lambda_j(t , x) 
  = \int_0^t  \frac{F(\diff s, j, x )}{F(s-, 0, x )},
  \quad j \in \{1,2\}.
\end{equation} Cause-specific risk predictions can be obtained from
\DIFdelbegin \DIFdel{\( \Lambda_1 \)
and $\Lambda_2$ using the formula
\mbox{%DIFAUXCMD
\citep[e.g.,][]{benichou1990estimates, ozenne2017riskregression}}\hspace{0pt}%DIFAUXCMD
,
}\begin{displaymath}
  \DIFdel{%DIFDELCMD < \label{eq:cs-risk-def}%%%
  Q(T \leq t, D = j \mid X=x) =
  \int_0^t \exp\left\{-\Lambda_{1}(u \mid x)-\Lambda_{2}(u
    \mid x)\right\}  \Lambda_j(\diff u \mid x),
  \quad j \in \{1,2\}.
}\end{displaymath}%DIFAUXCMD
\DIFdel{Hence, given }\DIFdelend the joint survival super learner \DIFdelbegin \DIFdel{'s estimate of \( F \) we
can use equation~}%DIFDELCMD < \eqref{eq:7} %%%
\DIFdel{to obtain estimates of the
cause-specific cumulative hazard functions $\Lambda_j$, which can in
turn be used to obtain estimates of the cause-specific risks through
equation~}%DIFDELCMD < \eqref{eq:cs-risk-def}%%%
\DIFdel{. For instance, in the prostate cancer
example, this expression will provide the conditional probability that
a patient with a certain set of baseline characteristics will,
before
time point \( t \), have had tumour recurrence or have died without
tumour recurrence.
}%DIFDELCMD < 

%DIFDELCMD < %%%
\DIFdel{We have suggested to implement }\DIFdelend \DIFaddbegin \eqref{eq:discrete-JSLL} \DIFadd{by
substituting into the well-known formula
\mbox{%DIFAUXCMD
\citep[e.g.,][]{benichou1990estimates, ozenne2017riskregression}}\hspace{0pt}%DIFAUXCMD
,
}\begin{equation}
  \DIFadd{\label{eq:cs-risk-def} \hat Q_n(T \leq t, D = j \mid X=x) = \int_0^t
\exp\left\{-\hat\Lambda_{1n}(u \mid x)-\Lambda_{2n}(u \mid
x)\right\} \hat\Lambda_{jn}(\diff u \mid x), \quad j \in \{1,2\}.
}\end{equation} \DIFadd{Furthermore, }\DIFaddend the joint survival super learner \DIFdelbegin \DIFdel{by building a library
using learners of the cause-specific cumulative hazard functions,
$\Lambda_j$, and the cumulative hazard function for censoring ,
$\Gamma$. With this implementation, we can directly input the highest
ranked tuple of cause-specific hazard functions
$(\Lambda_1, \Lambda_2)$ provided by the joint survival super learner as input to
equation~}%DIFDELCMD < \eqref{eq:cs-risk-def}%%%
\DIFdel{.
}\DIFdelend \DIFaddbegin \DIFadd{provides an
estimator of the censoring distribution:
}\begin{equation*}
 \DIFadd{\hat G_n(T \leq t \mid X=x) = \exp\left\{-\hat\Gamma_n(t \mid x)\right\}.
}\end{equation*}

%DIF >  Hence, given the joint survival super learner's
%DIF >  estimate of \( F \) we can use equation~\eqref{eq:7} to obtain
%DIF >  estimates of the cause-specific cumulative hazard functions
%DIF >  $\Lambda_j$, which can in turn be used to obtain estimates of the
%DIF >  cause-specific risks through equation~\eqref{eq:cs-risk-def}. For
%DIF >  instance, in the prostate cancer example, this expression will provide
%DIF >  the conditional probability that a patient with a certain set of
%DIF >  baseline characteristics will, before time point \( t \), have had
%DIF >  tumour recurrence or have died without tumour recurrence.

%DIF >  We suggest to implement the joint survival super learner by building a
%DIF >  library using learners of the cause-specific cumulative hazard
%DIF >  functions, $\Lambda_j$, and the cumulative hazard function for
%DIF >  censoring, $\Gamma$. With this implementation, we can directly input
%DIF >  the highest ranked tuple of cause-specific hazard functions
%DIF >  $(\Lambda_1, \Lambda_2)$ provided by the joint survival super learner
%DIF >  as input to equation~\eqref{eq:cs-risk-def}.
\DIFaddend 

\section{Theoretical guarantees}
\label{sec:theor-results-prop}

\DIFdelbegin \DIFdel{The use of cross-validation underlying the joint survival super learner is }\DIFdelend \DIFaddbegin \DIFadd{Cross-validation is the backbone of super learning and }\DIFaddend an intuitively
reasonable procedure for fair model selection without overfitting. In
this section, we \DIFdelbegin \DIFdel{follow the works }\DIFdelend \DIFaddbegin \DIFadd{adapt the work }\DIFaddend of \cite{van2003unicv} and
\cite{vaart2006oracle} \DIFdelbegin \DIFdel{to }\DIFdelend \DIFaddbegin \DIFadd{and }\DIFaddend provide a theoretical justification for \DIFdelbegin \DIFdel{this practice }\DIFdelend \DIFaddbegin \DIFadd{the
joint survival super learner }\DIFaddend in the form of a finite-sample oracle
inequality. We begin by demonstrating that minimizing the integrated
Brier score, as defined in
Section~\DIFdelbegin \DIFdel{\ref{sec:super-learner-simple}}\DIFdelend \DIFaddbegin \DIFadd{\ref{sec:joint-survival-super-learner}}\DIFaddend , is statistically
\DIFdelbegin \DIFdel{meaningful, in that perfect }\DIFdelend \DIFaddbegin \DIFadd{proper, in the sense that }\DIFaddend minimisation recovers the parameter of the
data-generating distribution. Together with our finite-sample oracle
inequality (Proposition~\ref{prop:oracle-prop} below), this implies
that the joint survival super learner is consistent when it is based
on a library that includes \DIFdelbegin \DIFdel{a }\DIFdelend \DIFaddbegin \DIFadd{at least one }\DIFaddend consistent learner. Another
consequence of our finite-sample oracle inequality is that the joint
survival super learner converges \DIFdelbegin \DIFdel{at }\DIFdelend (nearly) \DIFaddbegin \DIFadd{at }\DIFaddend the optimal rate
achievable within the library \DIFaddbegin \DIFadd{of learners}\DIFaddend . This statement is made
precise in Corollary~\ref{cor:asymp-cons} and the following
discussion. Proofs are deferred to the Appendix.

A sensible loss function should attain the minimal expected value at
the parameter corresponding to the data-generating distribution. Loss
functions with this property are \DIFdelbegin \DIFdel{known as properscoring rules, and as
strictly proper scoring rules }\DIFdelend \DIFaddbegin \DIFadd{called proper, and strictly proper }\DIFaddend if
the minimizer is unique \citep{gneiting2007strictly}. Absence of
properness makes it unclear why minimizing the (estimated) expected
loss is interesting.  Proposition~\ref{prop:stric-prop} \DIFdelbegin \DIFdel{is a formal statement of the fact
}\DIFdelend \DIFaddbegin \DIFadd{states }\DIFaddend that
the integrated Brier score \DIFdelbegin \DIFdel{, }\DIFdelend as defined in
\DIFdelbegin \DIFdel{our setting (c.f.,
Section~\ref{sec:super-learner-simple}), }\DIFdelend \DIFaddbegin \DIFadd{Section~\ref{sec:joint-survival-super-learner} }\DIFaddend is a strictly proper
scoring rule. To \DIFdelbegin \DIFdel{state }\DIFdelend \DIFaddbegin \DIFadd{establish }\DIFaddend this result, recall that the function \(F\)
implicitly depends on the data-generating probability measure
\(P\in\mathcal P\) but that this was \DIFaddbegin \DIFadd{so-far }\DIFaddend suppressed in the
notation. We now make this dependence explicit by writing \(F_P\) for
the function determined by a given \(P \in\mathcal{P}\) in accordance
with equation equation~(\ref{eq:F-def}). In the following we let \(
\mathcal{H}_{\mathcal{P}} = \{F_P : P \in \mathcal{P}\} \).

\begin{proposition}
  \label{prop:stric-prop}
  If \(P \in\mathcal{P}\) then
  \begin{equation*}
    F_P = \argmin_{F \in \mathcal{H}_{\mathcal{P}}}
    \E_P{[\bar{B}_\tau(F, O)]}
    ,
  \end{equation*}
  for all \( l \in \{-1, 0, 1, 2 \} \), almost all
  \( t \in [0,\tau] \), and \( P \)-almost all
  \( x \in \mathcal{X} \).
\end{proposition}

\DIFaddbegin \DIFadd{The discrete joint survival super learner defined in
}\eqref{eq:discrete-JSLL} \DIFadd{provides an estimate of the function \(F\)
}\begin{equation*}
  \DIFadd{\hat{\phi}_n=\phi_{\hat \Lambda_{1n},\hat \Lambda_{2n}, \hat \Gamma_{n}}
}\end{equation*} \DIFadd{which is obtained by substituting \((\hat
\Lambda_{1n},\hat \Lambda_{2n}, \hat \Gamma_{n})\) for \((a_1,a_2,b)\) into the structural
equations }\eqref{eq:anti-transition}\DIFadd{. }\DIFaddend To evaluate the performance of
\DIFdelbegin \DIFdel{the joint survival super learner we might }\DIFdelend \DIFaddbegin \DIFadd{\(\hat{\phi}_n\) we }\DIFaddend benchmark it against the data-generating
\DIFaddbegin \DIFadd{distribution }\DIFaddend \( F_P \), \DIFdelbegin \DIFdel{as this has }\DIFdelend \DIFaddbegin \DIFadd{which according to
Proposition~\ref{prop:stric-prop} has the }\DIFaddend smallest expected
loss\DIFdelbegin \DIFdel{by Proposition~\ref{prop:stric-prop}. A more
nuanced comparison is to benchmark it against }\DIFdelend \DIFaddbegin \DIFadd{. Another useful theoretical benchmark is the so-called oracle
learner which is }\DIFaddend the best learner \DIFdelbegin \DIFdel{available given the library and the training data. This is the
so-called oracle learner, formally defined as
}\DIFdelend \DIFaddbegin \DIFadd{included in the library of learners
and formally defined by
}\DIFaddend \begin{equation*}
  \tilde{\phi}_n
  =  \argmin_{\phi \in \mathcal{F}(\mathcal{A}_1, \mathcal{A}_2, \mathcal{B}) }
  \tilde{R}_{n}(\phi ; \bar{B}_{\tau}),
  \quad \text{with} \quad 
  \tilde{R}_n(\phi; \bar{B}_{\tau})=
  \frac{1}{K}\sum_{k=1}^{K} 
  \E_P{
    \left[
      \bar{B}_{\tau}
      {
        \left(
          \phi{ (\data_n^{-k})}
          , O
        \right)
      } 
      \midd  \data_n^{-k}
    \right]}
  ,
\end{equation*}
where %DIF <  \(  \mathcal{F}(\mathcal{A}_1, \mathcal{A}_2, \mathcal{B}) \)
%DIF <  was defined in equation~\eqref{eq:jssl-lib-def}, and 
we use \( \E_P \) to denote \DIFaddbegin \DIFadd{the }\DIFaddend expectation
under the distribution \( P \) for a new observation \( O \) \DIFdelbegin \DIFdel{independent of }\DIFdelend \DIFaddbegin \DIFadd{which is
independent of the data }\DIFaddend \( \data_n^{-k} \). Like the joint survival
super learner, the oracle learner depends on the library of learners
and on the \DIFdelbegin \DIFdel{specific datapartitions}\DIFdelend \DIFaddbegin \DIFadd{actual partition of the data}\DIFaddend , but unlike the joint survival
super learner, it also depends on the unknown data-generating
distribution. It is hence not available in practice and serves only as
a theoretical benchmark.

In the following, we equip the space \( \mathcal{H}_{\mathcal{P}} \)
with the norm
\begin{equation}
  \label{eq:norm}
  \| F \|_{P} = 
  \left\{
    \sum_{l=-1}^{2}
    \int_0^{\tau} \E_P{\left[ F(t, l, X)^2 \right]} \diff t
  \right\}^{1/2}.
\end{equation} This norm is \DIFaddbegin \DIFadd{a natural performance which is }\DIFaddend equal to
the excess risk\DIFaddbegin \DIFadd{, }\DIFaddend \( \E_P{[\bar{B}_\tau(F, O)]} -
\E_P{[\bar{B}_\tau(F_P, O)]} \)\DIFdelbegin \DIFdel{by
}\DIFdelend \DIFaddbegin \DIFadd{, as shown in }\DIFaddend Lemma~\ref{lemma:norm} in
the Appendix\DIFdelbegin \DIFdel{and is a natural performance
measure}\DIFdelend . For simplicity of presentation, we \DIFdelbegin \DIFdel{take \( n \) and the data partitions to be such that }\DIFdelend \DIFaddbegin \DIFadd{assume that all folds
of the data partition have equal size, }\DIFaddend \( |\data_n^{-k}| = n/K \) \DIFdelbegin \DIFdel{with \( K \)
fixed . We will }\DIFdelend \DIFaddbegin \DIFadd{for
a fixed number of folds \( K \). We }\DIFaddend allow the number of learners
to grow with \( n \) and write \DIFdelbegin \DIFdel{\( \mathcal{F}_n=\mathcal{F}(\mathcal{A}_{1,n}, \mathcal{A}_{2,n},
\mathcal{B}_n)\) }\DIFdelend \DIFaddbegin \DIFadd{\(
\mathcal{F}_n=\mathcal{F}(\mathcal{A}_{1n}, \mathcal{A}_{2n},
\mathcal{B}_n)\) }\DIFaddend as short-hand notation emphasing the dependence on
the sample size. We now state a finite-sample inequality that bounds
the performance of the joint survival super learner relative to that
of the oracle learner.

\begin{proposition}
  \label{prop:oracle-prop}
  For all \(P\in\mathcal{P}\), \( n \in \N \), \( k \in \{1, \dots, K\} \),
  and $\delta>0$,
  \begin{align*}
    \frac{1}{K}\sum_{k=1}^{K} \E_{P}{\left[ \Vert \hat{\phi}_n(\data_n^{-k}) - F_P \Vert_{P}^2 \right]}
    & \leq (1 + 2\delta)
      \frac{1}{K}\sum_{k=1}^{K} \E_{P}{\left[ \Vert \tilde{\phi}_n(\data_n^{-k}) - F_P \Vert_{P}^2 \right]}
    \\
    & \quad
      + (1+ \delta) 16   K \tau
      \left(
      13 + \frac{12}{\delta}
      \right)
      \frac{\log(1 + |\mathcal{F}_n|)}{n}.
  \end{align*}
\end{proposition}

\DIFdelbegin \DIFdel{The expectations }\DIFdelend \DIFaddbegin \DIFadd{Note that the expectation }\DIFaddend in Proposition~\ref{prop:oracle-prop} \DIFdelbegin \DIFdel{reflect a mild
abuse of notation, in that they are formally }\DIFdelend \DIFaddbegin \DIFadd{is
}\DIFaddend taken with respect to the product measure \( P^{n} \) for the \DIFdelbegin \DIFdel{whole }\DIFdelend data set
\( \data_n \). This means that we are quantifying the average
performance of the joint survival super learner across \DIFdelbegin \DIFdel{average training
data }\DIFdelend \DIFaddbegin \DIFadd{all training
data of size \(n\). A corresponding quantity was called the expected
true error rate in \mbox{%DIFAUXCMD
\cite{efron_and_tibshirani97}}\hspace{0pt}%DIFAUXCMD
}\DIFaddend . As with many
finite-sample oracle inequalities, this result is of little direct
practical utility because the right-hand side depends on
data-dependent, unknown quantities. However, it does quantify how the
number of folds, the time horizon, and the number of learners in the
library can be expected to influence the performance. The result has
the following asymptotic consequences.

\begin{corollary}
  \label{cor:asymp-cons}
  Assume that \( |\mathcal{F}_n| = \bigO(n^q)\), for some
  \( q \in \N \) and that there exists a sequence
  \( \phi_n \in \mathcal{F}_n \), \( n \in \N \), such that
  \(  \E_{P}{\left[ \Vert
      \phi_n(\data_n^{-k}) - F_P \Vert_{P}^2 \right]} = C_P +
  \bigO(n^{-\alpha}) \), for some \( \alpha\leq 1 \) and
  \( C_P \geq 0 \).
  \begin{enumerate}[label=(\alph*)]
  \item\label{item:1} If $\alpha=1$, then
    \(\frac{1}{K}\sum_{k=1}^{K} \E_{P}{\left[ \Vert
        \hat{\phi}_n(\data_n^{-k}) - F_P \Vert_{P}^2 \right]} = C_P +
    \bigO(\log(n)^{1+\epsilon}n^{-1}) \), $\forall\epsilon>0$.
  \item\label{item:2} If $\alpha<1$, then
    \(\frac{1}{K}\sum_{k=1}^{K} \E_{P}{\left[ \Vert
        \hat{\phi}_n(\data_n^{-k}) - F_P \Vert_{P}^2 \right]} = C_P +
    \bigO(n^{-\alpha}) \).
  \end{enumerate}
\end{corollary}

Proposition~\ref{prop:oracle-prop} \DIFdelbegin \DIFdel{provided a precise }\DIFdelend \DIFaddbegin \DIFadd{thus provides a }\DIFaddend finite-sample bound
on the average price we pay for using cross-validation, and
Corollary~\ref{cor:asymp-cons} states that this price vanishes
asymptotically, up to poly-logarithmic terms, provided that the size
of the library does not grow faster than \DIFaddbegin \DIFadd{at }\DIFaddend a polynomial rate in the
sample size. The situation \( C_P=0 \) corresponds to a setting in
which the library includes a consistent learner. Cases~\ref{item:1}
and~\ref{item:2} correspond to situations where the oracle learner
achieves a parametric or non-parametric asymptotic rate of
convergence, respectively.

To illustrate the content of Corollary~\ref{cor:asymp-cons}, consider
first a situation where we use a library with an increasing number of
\DIFdelbegin \DIFdel{semi-parametric Cox models with different interaction terms, as well
as several Poisson regression modelsbased on different
discretisations of the time scale}\DIFdelend \DIFaddbegin \DIFadd{Cox regression models}\DIFaddend . Each of these models will \DIFdelbegin \DIFdel{independently }\DIFdelend achieve a parametric
rate of convergence, \DIFaddbegin \DIFadd{to possibly different least-false cumulative
hazard functions \mbox{%DIFAUXCMD
\cite{hjort1992inference}}\hspace{0pt}%DIFAUXCMD
, }\DIFaddend and hence
item~\ref{item:1} of Corollary~\ref{cor:asymp-cons} states that the
joint survival super learner based on this library will achieve a
near-parametric rate of convergence.  \DIFdelbegin \DIFdel{The constant }\DIFdelend \( C_P \) can be \DIFdelbegin \DIFdel{taken equal to the distance to }\DIFdelend \DIFaddbegin \DIFadd{set to be the
distance between the data-generating distribution and }\DIFaddend the least false
model in the library, and so the joint survival super learner will
approximate the least false model in the library at a near-parametric
rate. Another situation appears if we add more flexible models to the
library, such as Cox lasso or random survival forests.  These models
typically converge at \DIFdelbegin \DIFdel{non-parametric rates, with the fastest rate depending on the unknown }\DIFdelend \DIFaddbegin \DIFadd{slower rates, where the fastest achievable rate
depends on the }\DIFaddend data-generating distribution.  Item~\ref{item:2} of
Corollary~\ref{cor:asymp-cons} shows that the joint survival super
learner achieves the same convergence rate as the best-performing
\DIFdelbegin \DIFdel{algorithm }\DIFdelend \DIFaddbegin \DIFadd{learner }\DIFaddend in the library, without any knowledge of the data-generating
distribution.

\section{Numerical experiments}
\label{sec:numer-exper}

\DIFdelbegin \DIFdel{In this section, we report results from a simulation study where we
consider estimation of the conditional survival function. We restrict
attention to the simpler setting without competing risks to enable
comparison with existing super learner methods. In the first part, we
compare }\DIFdelend \DIFaddbegin \DIFadd{The numerical experiments have two aims. The first aim is to
demonstrate that }\DIFaddend the joint survival super learner \DIFdelbegin \DIFdel{to two }\DIFdelend \DIFaddbegin \DIFadd{can outperform the
}\DIFaddend IPCW based discrete super learners \DIFdelbegin \DIFdel{that use either the Kaplan-Meier estimator or a Cox
model to estimate the censoring
probability
\mbox{%DIFAUXCMD
\citep{gonzalez2021stacked}}\hspace{0pt}%DIFAUXCMD
. In the second part, we compare the }\DIFdelend \DIFaddbegin \DIFadd{of \mbox{%DIFAUXCMD
\citep{gonzalez2021stacked}
}\hspace{0pt}%DIFAUXCMD
which prespecify a potentially misspecified model for the censoring
mechanism. The second aim is to show that the discrete }\DIFaddend joint survival
super learner \DIFdelbegin \DIFdel{to the }\DIFdelend \DIFaddbegin \DIFadd{can compete and outperform the ensemble }\DIFaddend super learner
proposed by \cite{westling2021inference}.

\DIFdelbegin \DIFdel{In both parts, we use the same data-generating mechanism. We generate data
according to a distribution motivated from a real data set in which censoring
depends on the baseline covariates. We simulate data based on }\DIFdelend \DIFaddbegin \DIFadd{For the numerical experiments we have synthesized }\DIFaddend the prostate cancer
\DIFdelbegin \DIFdel{study of \mbox{%DIFAUXCMD
\cite{kattan2000pretreatment} }\hspace{0pt}%DIFAUXCMD
}\DIFdelend \DIFaddbegin \DIFadd{data of \mbox{%DIFAUXCMD
\cite{kattan2000pretreatment} }\hspace{0pt}%DIFAUXCMD
by fitting a hierarchical
structural equation model under parametric assumptions}\DIFaddend . The outcome of
interest is the time \DIFdelbegin \DIFdel{to tumour recurrence , and five }\DIFdelend \DIFaddbegin \DIFadd{from randomization until the combined endpoint
tumour recurrence or all-cause death. Five }\DIFaddend baseline covariates are
used to predict \DIFdelbegin \DIFdel{outcome}\DIFdelend \DIFaddbegin \DIFadd{the outcome risk}\DIFaddend : prostate-specific antigen (PSA,
ng/mL), Gleason score sum (GSS, values between 6 and 10), radiation
dose (RD), hormone therapy (HT, yes/no) and clinical stage (CS, six
values). The study was designed such that a patient's radiation dose
depended on when the patient entered the study\DIFdelbegin \DIFdel{\mbox{%DIFAUXCMD
\citep{gerds2013estimating}}\hspace{0pt}%DIFAUXCMD
}\DIFdelend . This in turn implies
that the time of censoring depends on the radiation dose. The data
were re-analysed in \citep{gerds2013estimating} where a sensitivity
analysis was conducted based on simulated data. Here we use the same
simulation setup, where event and censoring times are generated
according to parametric Cox-Weibull models \DIFaddbegin \DIFadd{\mbox{%DIFAUXCMD
\citep{Bender2005}
}\hspace{0pt}%DIFAUXCMD
}\DIFaddend estimated from the original data, and the covariates are generated
according to either marginal Gaussian normal or binomial distributions
estimated from the original data
\citep[c.f.,][Section~4.6]{gerds2013estimating}. We refer to this
simulation setting as `dependent censoring'. We also considered a
simulation setting where data were generated in the same way, except
that censoring was generated completely independently \DIFaddbegin \DIFadd{of the
covariates}\DIFaddend . We refer to this simulation setting as `independent
censoring'.

For all super learners, we use a library consisting of three learners:
The \DIFdelbegin \DIFdel{Kaplan-Meier estimator \mbox{%DIFAUXCMD
\citep{kaplan1958nonparametric,Gerds_2019prodlim}}\hspace{0pt}%DIFAUXCMD
}\DIFdelend \DIFaddbegin \DIFadd{Nelson-Aalen estimator \mbox{%DIFAUXCMD
\citep{andersen2012statistical}}\hspace{0pt}%DIFAUXCMD
}\DIFaddend , a Cox
\DIFdelbegin \DIFdel{model
with main effects \mbox{%DIFAUXCMD
\citep{cox1972regression, survival-package}}\hspace{0pt}%DIFAUXCMD
}\DIFdelend \DIFaddbegin \DIFadd{regression model with additive effects of the covariates
\mbox{%DIFAUXCMD
\citep{cox1972regression}}\hspace{0pt}%DIFAUXCMD
}\DIFaddend , and a random survival forest
\DIFdelbegin \DIFdel{\mbox{%DIFAUXCMD
\citep{ishwaran2008random,randomForestSRC}}\hspace{0pt}%DIFAUXCMD
}\DIFdelend \DIFaddbegin \DIFadd{\mbox{%DIFAUXCMD
\citep{ishwaran2008random}}\hspace{0pt}%DIFAUXCMD
}\DIFaddend . We use the same library to learn the
\DIFdelbegin \DIFdel{outcome distribution and the censoring
distribution. The three learners in our library of
learners can be used to learn the }\DIFdelend cumulative hazard functions of the outcome and the censoring \DIFdelbegin \DIFdel{distribution. The latter
works by training the learner on the data set \( \data_n^c \),
where \( \data_n^c = \{O_i^c\}_{i=1}^n \) with
\( O_i^c = (\tilde{T}_i, 1-\Delta_i, X_i) \). When we say that
we use a learner for the cumulative
hazard function of the outcome to learn the cumulative hazard function of the censoring time, we mean that the learner is trained on
\( \data_n^c \)}\DIFdelend \DIFaddbegin \DIFadd{time,
respectively. Specifically, to obtain estimates of the cumulative
censoring hazard function we fit the learners to the modified data set
\(\{(\tilde{T}_i, 1-\Delta_i, X_i)\}_{i=1}^n \) where the roles of
censoring and outcome are exchanged}\DIFaddend .

We compare the joint survival super learner to two IPCW based super
learners: The first super learner, called IPCW(Cox), uses a Cox
\DIFdelbegin \DIFdel{model with main
effects }\DIFdelend \DIFaddbegin \DIFadd{regression model with additive effects of the five covariates }\DIFaddend to
estimate the censoring probabilities, while the second super learner,
called IPCW(KM), uses the Kaplan-Meier estimator to estimate the
censoring probabilities. \DIFdelbegin \DIFdel{The }\DIFdelend \DIFaddbegin \DIFadd{Note that the }\DIFaddend Cox model for the censoring
distribution is correctly specified in both simulation settings while
the Kaplan Meier estimator only estimates the censoring model
correctly in the simulation setting where censoring is
independent. Both IPCW super learners are fitted using the
\texttt{R}-package \texttt{riskRegression}
\citep{Gerds_Ohlendorff_Ozenne_2023}.
%
% | time | sim_setting | true_events | true_cens | at_risk |
% |------+-------------+-------------+-----------+---------|
% |   36 | original    |      24.619 |    61.853 |  25.774 |
% |   36 | indep_cens  |      24.674 |    38.740 |  46.141 |
%
The IPCW super learners use the integrated Brier score up to a fixed time
horizon (36 months). The marginal risk of the event before this time horizon is
\(\approx 24.6\)\%. Under the `dependent censoring' setting the marginal
censoring probability before the time horizon is \(\approx 61.9\)\%. Under the
`independent censoring' setting the marginal censoring probability before this
time horizon is \( \approx 38.7 \)\%.

Each super learner provides a learner for the cumulative hazard
function for the outcome of interest. From the cumulative hazard
function, we obtain a risk prediction model as described in
Section~\DIFdelbegin \DIFdel{\ref{sec:use-cases-state}}\DIFdelend \DIFaddbegin \DIFadd{\ref{sec:joint-survival-super-learner}}\DIFaddend , with the special case
of $\Lambda_2 = 0$. We measure the performance of each super learner
by calculating the index of prediction accuracy (IPA)
\citep{kattan2018index} at a fixed time horizon (36 months) for the
risk prediction model provided by the super learner. The IPA is 1
minus the ratio between the model's Brier score and the null model's
Brier score, where the null model is the model that does not use any
covariate information. The \DIFaddbegin \DIFadd{value of }\DIFaddend IPA is approximated using a large
(\( n = 20,000 \)) independent data set of uncensored data. As a
benchmark, we calculate the performance of the risk prediction model
chosen by the oracle selector, which \DIFdelbegin \DIFdel{uses the large data set of
uncensored event times to select the learner with }\DIFdelend \DIFaddbegin \DIFadd{has }\DIFaddend the highest IPA \DIFaddbegin \DIFadd{in the
simulation setting}\DIFaddend .

The results \DIFaddbegin \DIFadd{for the first aim }\DIFaddend are shown in
Figure~\ref{fig:ipcw-fail}. We see that in the scenario where
censoring depends on the covariates, using the Kaplan-Meier estimator
to estimate the censoring probabilities provides a risk prediction
model with an IPA that is lower than the risk prediction model
provided by the joint survival super learner. The performance of the
risk prediction model selected by the joint survival super learner is
similar to the risk prediction model selected by the IPCW(Cox) super
learner which a priori uses a correctly specified model for the
censoring distribution. Both these risk prediction models are close to
the performance of the oracle, except for small sample sizes.

\begin{figure}
\figuresize{0.7}
\figurebox{20pc}{25pc}{}[experiment-fig-sl-ipcw]
\caption{For the risk prediction models provided by each of the
    super learners, the IPA is plotted against sample size. The
    results are averages across 1000 simulated data sets and the error
    bars are used to quantify the Monte Carlo uncertainty. JSSL
    denotes the joint survival super learner. }
\label{fig:ipcw-fail}
\end{figure}

\DIFdelbegin \DIFdel{We next compare the joint survival super learner to }\DIFdelend \DIFaddbegin \DIFadd{For the second aim, we consider }\DIFaddend the super learner \DIFdelbegin \DIFdel{survSL \mbox{%DIFAUXCMD
\citep{westling2021inference}}\hspace{0pt}%DIFAUXCMD
. This is another super learner
}\DIFdelend \DIFaddbegin {\it \DIFadd{survSL}}
\DIFadd{proposed by \mbox{%DIFAUXCMD
\citep{westling2021inference} }\hspace{0pt}%DIFAUXCMD
}\DIFaddend which like the joint
survival super learner \DIFdelbegin \DIFdel{works without }\DIFdelend \DIFaddbegin \DIFadd{does not require }\DIFaddend a pre-specified censoring
model. Both \DIFdelbegin \DIFdel{the joint survival super learner
and survSL }\DIFdelend \DIFaddbegin \DIFadd{methods }\DIFaddend provide estimates of the event-time and censoring
distributions \DIFdelbegin \DIFdel{. Hence, we compare the performance of these methods }\DIFdelend \DIFaddbegin \DIFadd{and hence we compare their performance }\DIFaddend with respect to
both the outcome and the censoring distribution. Again we use the IPA
to quantify the predictive performance.

The results \DIFaddbegin \DIFadd{for the second aim }\DIFaddend are shown in
Figures~\ref{fig:zelefski-out} and~\ref{fig:zelefski-cens}. We see
that for most sample sizes, the joint survival super learner \DIFdelbegin \DIFdel{selects prediction models for both
censoring and outcome which have }\DIFdelend \DIFaddbegin \DIFadd{has
}\DIFaddend similar or higher IPA compared to \DIFdelbegin \DIFdel{the
prediction models selected by survSL .
}\DIFdelend \DIFaddbegin \DIFadd{survSL with respect to both the
prediction of the censoring and the outcome risks. The advantage of
the joint survival super learner in this particular simulation setting
seems to be mostly due that it is a discrete super learner whereas
survSL combines the learners.
}

\DIFaddend \begin{figure}
\figuresize{0.7}
\figurebox{20pc}{25pc}{}[experiment-fig-sl-survSL-out]
\caption{For the risk prediction models of the outcome provided by
  each of the super learners, the IPA at the fixed time horizon is
  plotted against sample size. The results are averages across 1000
  repetitions and the error bars are used to quantify the Monte Carlo
  uncertainty. JSSL denotes the joint survival super learner.}
\label{fig:zelefski-out}
\end{figure}

\begin{figure}
\figuresize{.7}
\figurebox{20pc}{25pc}{}[experiment-fig-sl-survSL-cens]
\caption{For the risk prediction models of the censoring model
    provided by each of the super learners, the IPA at the fixed time
    horizon is plotted against sample size. The results are averages
    across 1000 repetitions and the error bars are used to quantify
    the Monte Carlo uncertainty. JSSL denotes the joint survival super
    learner.}
\label{fig:zelefski-cens}
\end{figure}


\section{Prostate cancer study}
\label{sec:real-data-appl}

We use the prostate cancer data of \cite{kattan2000pretreatment} to
illustrate the use of the joint survival super learner in the presence
of competing risks. We have introduced the data in
Section~\ref{sec:numer-exper}. The data \DIFdelbegin \DIFdel{consists }\DIFdelend \DIFaddbegin \DIFadd{consist }\DIFaddend of 1,042 patients who
are followed from start of followup until tumour recurrence \DIFaddbegin \DIFadd{(n=268)}\DIFaddend ,
death without tumour recurrence \DIFdelbegin \DIFdel{, or end of followup (censored }\DIFdelend \DIFaddbegin \DIFadd{(n=29), or censored (n=745}\DIFaddend ), whatever
came first. We use the joint survival super learner to rank libraries
of learners for the cause-specific cumulative hazard functions of
tumour recurrence, death without tumour recurrence, and censoring. The
\DIFaddbegin \DIFadd{three }\DIFaddend libraries of learners each include five learners: the
Nelson-Aalen estimator, three Cox regression models (unpenalized,
lasso, elastic net) each including additive effects of the baseline
covariates, and a random survival forest. \DIFdelbegin \DIFdel{We use the same set of learners to estimate
the cumulative hazard function of }\DIFdelend \DIFaddbegin \DIFadd{The Nelson-Aalen estimator
is estimated without covariates and serves as a benchmark model which
guarantees that the joint survival super learner is not worse than a
model which predicts the same probability to all individuals. The
other 4 learners use the 5 baseline covariates listed in section
\ref{sec:numer-exper} to predict the three cumulative hazard functions
of time to }\DIFaddend tumour recurrence \( \Lambda_1 \), \DIFdelbegin \DIFdel{the cumulative hazard function of }\DIFdelend \DIFaddbegin \DIFadd{time to }\DIFaddend death without
tumour recurrence \( \Lambda_2 \), and \DIFdelbegin \DIFdel{the cumulative hazard function of the conditional
censoring distribution }\DIFdelend \DIFaddbegin \DIFadd{time to censoring }\DIFaddend $\Gamma$. \DIFdelbegin \DIFdel{This gives a library consisting }\DIFdelend \DIFaddbegin \DIFadd{The
resulting library consists }\DIFaddend of \( 5^3 = 125 \) learners\DIFdelbegin \DIFdel{for the conditional state occupation
probability function \( F \) defined in equation~(\ref{eq:F-def})}\DIFdelend . We use five
folds for training and testing the models, \DIFdelbegin \DIFdel{and we }\DIFdelend repeat training and
evaluation five times with different splits\DIFdelbegin \DIFdel{. The
}\DIFdelend \DIFaddbegin \DIFadd{, and obtain the discrete
joint survival super learner as the combination with the best average
5-fold }\DIFaddend integrated Brier score \DIFdelbegin \DIFdel{(}\DIFdelend \DIFaddbegin \DIFadd{across the 5 repetitions. Table
\ref{tab:1} shows integrated Brier scores for the conditional state
occupation probability function \( F \) as }\DIFaddend defined in
Section~\DIFdelbegin \DIFdel{\ref{sec:super-learner-simple}) for all learnersare shown in
Figure~\ref{fig:zelefski-real}}\DIFdelend \DIFaddbegin \DIFadd{\ref{sec:joint-survival-super-learner} evaluated 3 years after
randomization for a selection of the 125 learners}\DIFaddend . We see that \DIFaddbegin \DIFadd{among
the 5 best combinations }\DIFaddend the \DIFdelbegin \DIFdel{prediction performance
is mostly affected by the choice of learner for the censoring
distribution. Several combinations of learners give similar
performance as measured by the integrated Brier score, provided that
}\DIFdelend random survival forest is \DIFdelbegin \DIFdel{used to model the censoring distribution. }%DIFDELCMD < 

%DIFDELCMD < \begin{figure}
%DIFDELCMD < \figuresize{.7}
%DIFDELCMD < \figurebox{20pc}{25pc}{}[real-data-state-learner]
%DIFDELCMD < %%%
%DIFDELCMD < \caption{%
{%DIFAUXCMD
\DIFdelFL{The results of applying the 125 combinations of learners to the
    prostate cancer data set. The error bars are based on five repetitions using
    different splits. We refer to learners of \( \Lambda_1 \), \( \Lambda_2 \),
    and $\Gamma$ as `Tumour learner', `Mortality learner', and `Censoring
    learner', respectively.}}
%DIFAUXCMD
%DIFDELCMD < \label{fig:zelefski-real}
%DIFDELCMD < \end{figure}
%DIFDELCMD < 

%DIFDELCMD < %%%
\section{\DIFdel{Discussion}}
%DIFAUXCMD
\addtocounter{section}{-1}%DIFAUXCMD
%DIFDELCMD < \label{sec:discussion}
%DIFDELCMD < 

%DIFDELCMD < %%%
\DIFdel{A major advantage of the }\DIFdelend \DIFaddbegin \DIFadd{always selected
for \(\Gamma\) and that the differences for different learners of
\(\Lambda_1\) and \(\Lambda_2\) are small. To illustrate the
comparative performance of the discrete }\DIFaddend joint survival super learner
\DIFdelbegin \DIFdel{is that the
performance of each combination of learners can be estimated without
additional nuisance parameters. A potential drawback of our approach
is that we are evaluating the loss of the learners on the level of
}\DIFdelend \DIFaddbegin \DIFadd{we also split the data randomly into a training set with \(n=658\)
individuals and a test set with the remaining \(n=384\) individuals.
We fit the discrete joint survival super learner (5 repetitions of
5-fold cross-validation) and for comparison prespecified
cause-specific Cox regression models to the training set. In the
training set the discrete joint survival super learner chooses the
unpenalized Cox regression model for tumor recurrence, }\DIFaddend the \DIFdelbegin \DIFdel{observed data distribution, while the
target of the analysis is either
the
event-time distribution, or the censoring distribution, or both. Specifically, the finite-sample oracle inequality in Proposition~\ref{prop:oracle-prop} concerns the function \( F \),
which is a feature of \( P \in \mathcal{P} \), while what we are
typically interested in is \( \Lambda_j \) or \( S \), which are
features of \( Q \in \mathcal{Q} \). We
emphasize that , while the }\DIFdelend \DIFaddbegin \DIFadd{elastic net
Cox regression for death without tumor recurrence and the random
survival forest for censoring. Figure \ref{fig:7} compares the 3-year
risk predictions in the test set. Figure \ref{fig:8} shows the
corresponding calibration plots for the 3-year risk predictions. We
see that the discrete }\DIFaddend joint survival super learner \DIFdelbegin \DIFdel{provides estimates of \( \Lambda_j \) and
$\Gamma$ based on libraries \( \mathcal{A}_j \) and \( \mathcal{B} \),
the performance of these learners is not assessed directly in terms of
their respective target parameters, but only indirectly through the performance of \( F \). For settings without competing risks, our
numerical studies suggest that measuring the performance of \( F \)
also leads to good performance for estimation of \( S \).
}%DIFDELCMD < 

%DIFDELCMD < %%%
\DIFdel{Our proposed super learner can be implemented with a broad library of
learners and using existing software. Furthermore, while the library
\( \mathcal{F}(\mathcal{A}_1,\mathcal{A}_2,\mathcal{B}) \) consists of
\( |\mathcal{A}_1||\mathcal{A}_2||\mathcal{B}| \) many learners, we
only need to fit
\( |\mathcal{A}_1| +|\mathcal{A}_2| + |\mathcal{B}| \) many learners
in each fold. To evaluate the performance of each learner, we need to
perform \( |\mathcal{A}_1||\mathcal{A}_2||\mathcal{B}| \) many
operations to calculate the integrated Brier score in each hold-out
sample (one for each combination of the fitted models), but these
operations are often negligible compared to fitting the models. Hence
the }\DIFdelend \DIFaddbegin \DIFadd{is competitive for
the prediction of tumor recurrence and death without tumor recurrence
(note that even though the AUC is higher for the prespecified Cox
regression model, the Brier score is lower for the discrete }\DIFaddend joint
survival super learner\DIFdelbegin \DIFdel{is essentially not more
computationally demanding than any procedure that uses super learning
to learn $\Lambda_1$, $\Lambda_2$, and $\Gamma$ separately. While our
proposal is based on constructing the library \( \mathcal{F} \) from
libraries for learning \( \Lambda_1 \), $\Lambda_2$, and $\Gamma$, it
could also be of interest to consider learners that estimate \( F \)
directly. }\DIFdelend \DIFaddbegin \DIFadd{), and much better at predicting the 3-year
probabilities.
}\DIFaddend 

\DIFdelbegin \DIFdel{In our numerical studies, we only considered learners of $\Lambda_j$ and
$\Gamma$ that provide cumulative hazard functions which are piecewise constant
in the time argument. This simplifies the calculation of \( F \) as the integrals in equation~(\ref{eq:transition}) reduce to sums. When $\Lambda_j$ or
\( \Gamma \) are absolutely continuous in the time argument, calculating \( F \)
is more involved, but we expect that a good approximation can be achieved by
discretisation. }\DIFdelend \DIFaddbegin \begin{table}[h]
  \caption{\DIFaddFL{The results of applying the 125 combinations of learners to
the prostate cancer data set. Shown are the best 5 combinations and
selected intermediate ranks.  The ``Loss'' is the integrated Brier
score evaluated at 36 months and ``SD'' is the standard deviation
across the 5 repetitions of 5-fold cross-validation. The discrete
joint survival super learner chooses rank 1. }}
\begin{tabular}{ l| c c c c c } 
 \DIFaddFL{Rank}&\DIFaddFL{Cause 1}&\DIFaddFL{Cause 2}&\DIFaddFL{Censored}&\DIFaddFL{Loss}&\DIFaddFL{SD }\\\hline
 \DIFaddFL{$   1 $}&\DIFaddFL{Elastic net}&\DIFaddFL{Elastic net}&\DIFaddFL{Random forest}&\DIFaddFL{$  7.0205 $}&\DIFaddFL{$ 0.030977 $ }\\
 \DIFaddFL{$   2 $}&\DIFaddFL{LASSO}&\DIFaddFL{Elastic net}&\DIFaddFL{Random forest}&\DIFaddFL{$  7.0209 $}&\DIFaddFL{$ 0.031035 $ }\\
 \DIFaddFL{$   3 $}&\DIFaddFL{Cox}&\DIFaddFL{Elastic net}&\DIFaddFL{Random forest}&\DIFaddFL{$  7.0224 $}&\DIFaddFL{$ 0.030871 $ }\\
 \DIFaddFL{$   4 $}&\DIFaddFL{Elastic net}&\DIFaddFL{LASSO}&\DIFaddFL{Random forest}&\DIFaddFL{$  7.0225 $}&\DIFaddFL{$ 0.030170 $ }\\
 \DIFaddFL{$   5 $}&\DIFaddFL{LASSO}&\DIFaddFL{LASSO}&\DIFaddFL{Random forest}&\DIFaddFL{$  7.0228 $}&\DIFaddFL{$ 0.030237 $ }\\
 \DIFaddFL{$  25 $}&\DIFaddFL{Random forest}&\DIFaddFL{Random forest}&\DIFaddFL{LASSO}&\DIFaddFL{$  7.3845 $}&\DIFaddFL{$ 0.026975 $ }\\
 \DIFaddFL{$  50 $}&\DIFaddFL{Elastic net}&\DIFaddFL{Random forest}&\DIFaddFL{LASSO}&\DIFaddFL{$  7.3974 $}&\DIFaddFL{$ 0.021290 $ }\\
 \DIFaddFL{$  75 $}&\DIFaddFL{LASSO}&\DIFaddFL{Cox}&\DIFaddFL{LASSO}&\DIFaddFL{$  7.4059 $}&\DIFaddFL{$ 0.024660 $ }\\
 \DIFaddFL{$ 100 $}&\DIFaddFL{Nelson-Aalen}&\DIFaddFL{Cox}&\DIFaddFL{Elastic net}&\DIFaddFL{$  7.8300 $}&\DIFaddFL{$ 0.016719 $ }\\
 \DIFaddFL{$ 125 $}&\DIFaddFL{Nelson-Aalen}&\DIFaddFL{Nelson-Aalen}&\DIFaddFL{Nelson-Aalen}&\DIFaddFL{$ 10.3298 $}&\DIFaddFL{$ 0.003289 $ }\\
\end{tabular}\label{tab:1}
\end{table}
\DIFaddend 

\DIFdelbegin \DIFdel{Our original motivation for developing }\DIFdelend \DIFaddbegin \DIFadd{ANDERS: figuren viser flere resultater (se opdateret targets og sl-make-figures.org)
man kunne sikkert vise noget andet eller bar ingenting. lad os snakke om det eller du bestemmer
jeg har ikke beskrevet endnu hvad figuren viser fordi jeg er usikker paa om du vil have den med. 
}

\begin{figure}
%DIF >  \figuresize{.7}
\figurebox{20pc}{25pc}{}[risks_zelefsky_train_test.pdf]
\caption{\DIFaddFL{Comparison of the discrete joint survival super learner and pre-specified Cox regression ...}}
\label{fig:zelefski-real}
\end{figure}


\section{\DIFadd{Discussion}}
\label{sec:discussion}

\DIFadd{ANDERS: jeg vil helst holde den her discussion kort for biometrika. men, noget af det udkommenterede
kunne jo rykkes op i teksten hvis du synes at det skal vaere med. en ting som kunne kommentere paa her
er discrete vs ensemble 
}


%DIF >  A major advantage of the joint survival super learner is that the
%DIF >  performance of each combination of learners can be estimated without
%DIF >  additional nuisance parameters. A potential drawback of our approach
%DIF >  is that we are evaluating the loss of the learners on the level of the
%DIF >  observed data distribution, while the target of the analysis is either
%DIF >  the event-time distribution, or the censoring distribution, or both.
%DIF >  Specifically, the finite-sample oracle inequality in
%DIF >  Proposition~\ref{prop:oracle-prop} concerns the function \( F \),
%DIF >  which is a feature of \( P \in \mathcal{P} \), while what we are
%DIF >  typically interested in is \( \Lambda_j \) or \( S \), which are
%DIF >  features of \( Q \in \mathcal{Q} \). We emphasize that, while the
%DIF >  joint survival super learner provides estimates of \( \Lambda_j \) and
%DIF >  $\Gamma$ based on libraries \( \mathcal{A}_j \) and \( \mathcal{B} \),
%DIF >  the performance of these learners is not assessed directly in terms of
%DIF >  their respective target parameters, but only indirectly through the
%DIF >  performance of \( F \). For settings without competing risks, our
%DIF >  numerical studies suggest that measuring the performance of \( F \)
%DIF >  also leads to good performance for estimation of \( S \).

%DIF >  Our proposed super learner can be implemented with a broad library of
%DIF >  learners and using existing software. Furthermore, while the library
%DIF >  \( \mathcal{F}(\mathcal{A}_1,\mathcal{A}_2,\mathcal{B}) \) consists of
%DIF >  \( |\mathcal{A}_1||\mathcal{A}_2||\mathcal{B}| \) many learners, we
%DIF >  only need to fit
%DIF >  \( |\mathcal{A}_1| +|\mathcal{A}_2| + |\mathcal{B}| \) many learners
%DIF >  in each fold. To evaluate the performance of each learner, we need to
%DIF >  perform \( |\mathcal{A}_1||\mathcal{A}_2||\mathcal{B}| \) many
%DIF >  operations to calculate the integrated Brier score in each hold-out
%DIF >  sample (one for each combination of the fitted models), but these
%DIF >  operations are often negligible compared to fitting the models. Hence
%DIF >  the joint survival super learner is essentially not more
%DIF >  computationally demanding than any procedure that uses super learning
%DIF >  to learn $\Lambda_1$, $\Lambda_2$, and $\Gamma$ separately. While our
%DIF >  proposal is based on constructing the library \( \mathcal{F} \) from
%DIF >  libraries for learning \( \Lambda_1 \), $\Lambda_2$, and $\Gamma$, it
%DIF >  could also be of interest to consider learners that estimate \( F \)
%DIF >  directly.

%DIF >  In our numerical studies, we only considered learners of $\Lambda_j$ and
%DIF >  $\Gamma$ that provide cumulative hazard functions which are piecewise constant
%DIF >  in the time argument. This simplifies the calculation of \( F \) as the
%DIF >  integrals in equation~(\ref{eq:transition}) reduce to sums. When $\Lambda_j$ or
%DIF >  \( \Gamma \) are absolutely continuous in the time argument, calculating \( F \)
%DIF >  is more involved, but we expect that a good approximation can be achieved by
%DIF >  discretisation.

\DIFadd{In this article we have introduced the joint survival super learner
and illustrated how it can be used to simultaneously predict the risk
of the outcome events as well as the censoring probability over time
using baseline covariates. Another application of }\DIFaddend the joint survival
super learner \DIFdelbegin \DIFdel{was for use }\DIFdelend \DIFaddbegin \DIFadd{is }\DIFaddend within the framework of targeted \DIFdelbegin \DIFdel{or }\DIFdelend \DIFaddbegin \DIFadd{learning
\mbox{%DIFAUXCMD
\citep{van2011targeted}}\hspace{0pt}%DIFAUXCMD
, also known as }\DIFaddend debiased machine learning
\DIFaddbegin \DIFadd{\mbox{%DIFAUXCMD
\citep{chernozhukov2018double}}\hspace{0pt}%DIFAUXCMD
, }\DIFaddend -- a general methodology that combines
flexible, data-adaptive estimation \DIFaddbegin \DIFadd{of nuisance parameters }\DIFaddend with
asymptotically valid inference for low-dimensional target parameters\DIFdelbegin \DIFdel{\mbox{%DIFAUXCMD
\citep{van2011targeted,chernozhukov2018double}}\hspace{0pt}%DIFAUXCMD
.
In settings with
right-censored competing risks, the relevant nuisance parameters often
include }\DIFdelend \DIFaddbegin \DIFadd{.
For example, the methods in \mbox{%DIFAUXCMD
\cite{van2003unified,rytgaard2022targeted}
}\hspace{0pt}%DIFAUXCMD
target the average treatment effect in a survival setting which
require estimates of }\DIFaddend the cause-specific \DIFdelbegin \DIFdel{and }\DIFdelend \DIFaddbegin \DIFadd{cumulative hazard functions
and the }\DIFaddend censoring cumulative hazard \DIFdelbegin \DIFdel{functions
\mbox{%DIFAUXCMD
\citep[e.g.,][]{van2003unified,rytgaard2022targeted}}\hspace{0pt}%DIFAUXCMD
}\DIFdelend \DIFaddbegin \DIFadd{function}\DIFaddend . The joint survival super
learner \DIFdelbegin \DIFdel{immediately provides estimates of }\DIFdelend \DIFaddbegin \DIFadd{can be used to estimate }\DIFaddend these nuisance parameters and is \DIFdelbegin \DIFdel{hence particularly }\DIFdelend \DIFaddbegin \DIFadd{more
generally }\DIFaddend well suited for targeted and debiased machine learning \DIFdelbegin \DIFdel{. We leave the study of the joint
survival super learner in the context of targeted and debiased machine
learning for a future paper}\DIFdelend \DIFaddbegin \DIFadd{with
right censored data}\DIFaddend . 

\appendix

\section*{Appendix}


Define
\( \bar{B}_{\tau,P}(F, o) = \bar{B}_{\tau}(F, o) -
\bar{B}_{\tau}(F_P, o) \) and
\( R_{P}(F) = \E_P{[\bar{B}_{\tau,P}(F, O)]} \), where the
integrated Brier score \( \bar{B}_{\tau} \) was defined in
Section~\DIFdelbegin \DIFdel{\ref{sec:super-learner-simple}}\DIFdelend \DIFaddbegin \DIFadd{\ref{sec:joint-survival-super-learner}}\DIFaddend . Recall the norm
\( \Vert \blank \Vert_{P}\) defined in
equation~(\ref{eq:norm}).

\begin{lemma}
  \label{lemma:norm}
  \( R_{P}(F) = \Vert F - F_P \Vert_{P}^2 \).
\end{lemma}
\begin{proof}
  For any \( t \in [0, \tau] \) and \( l\in \{-1,0,1,2\} \) we have
  \begin{align*}
    & \E_{P}{\left[ (F(t, l, X) - \1{\{\eta(t) = l \}})^2 \right]}
    \\
    & =    \E_{P}{\left[ (F(t, l, X) - F_P(t, l, X) + F_P(t, l, X) - \1{\{\eta(t) = l
      \}})^2 \right]}
    \\
    & =    \E_{P}{\left[ (F(t, l, X) - F_P(t, l, X))^2\right]}
      + \E_{P}{\left[ (F_P(t, l, X) - \1{\{\eta(t) = l \}})^2\right]}
    \\
    & \quad
      + 2\E_{P}{\left[ (F(t, l, X) - F_P(t, l, X))(F_P(t, l, X) - \1{\{\eta(t) = l
      \}})\right]}
    \\
    & =    \E_{P}{\left[ (F(t, l, X) - F_P(t, l, X))^2\right]}
      + \E_{P}{\left[ (F_P(t, l, X) - \1{\{\eta(t) = l \}})^2\right]},
  \end{align*}
  where the last equality follows from the tower property. Hence, using Fubini,
  we have
  \begin{equation*}
    \E_P{[\bar{B}_{\tau}(F, O)]}
    = \Vert F - F_P \Vert_{P}^2 + \E_P{[\bar{B}_{\tau}(F_P, O)]}.
  \end{equation*}
\end{proof}

\begin{proof}[of Proposition~\ref{prop:stric-prop}]
  The result follows from Lemma~\ref{lemma:norm}.
\end{proof}

Recall that we use \( \mathcal{F}_n \) to denote a library of learners for the
function \( F \), and that \( \hat{\phi} \) and \( \tilde{\phi} \) denotes,
respectively, the discrete super learner and the oracle learner for the library
\( \mathcal{F}_n \), c.f., Section~\DIFdelbegin \DIFdel{\ref{sec:super-learner-simple}}\DIFdelend \DIFaddbegin \DIFadd{\ref{sec:joint-survival-super-learner}}\DIFaddend .

\begin{proof}[of Proposition~\ref{prop:oracle-prop}]
  Minimizing the loss \( \bar{B}_{\tau} \) is equivalent to
  minimizing the loss \( \bar{B}_{\tau,P} \), so the discrete super learner and
  oracle according to \( \bar{B}_{\tau} \) and \( \bar{B}_{\tau,P} \) are
  identical. By Lemma~\ref{lemma:norm}, \( R_P(F) \geq 0 \) for any
  \( F \in \mathcal{H}_{\mathcal{P}} \), and so using Theorem 2.3 from
  \citep{vaart2006oracle} with \( p=1 \), we have that for all \( \delta >0 \),
\begin{align*}
  & \frac{1}{K} \sum_{k=1}^{K} \E_{P}{\left[ R_P(\hat{\phi}_n(\data_n^{-k})) \right]}
  \\
  &  \quad \leq
    (1+2\delta)\frac{1}{K} \sum_{k=1}^{K}\E_{P}{\left[ R_P(\tilde{\phi}_n(\data_n^{-k})) \right]}
  \\
  & \qquad + (1+\delta) \frac{16 K}{n}
    \log(1 + |\mathcal{F}_n|)\sup_{F \in \mathcal{H}_{\mathcal{P}}}
    \left\{
    M(F) + \frac{v(F)}{R_P(F)}
    \left(
    \frac{1}{\delta} + 1
    \right)
    \right\}
\end{align*}
where for each \( F \in \mathcal{H}_{\mathcal{P}} \),
\( (M(F), v(F)) \) is some Bernstein pair for the function
\(o \mapsto \bar{B}_{\tau,P}(F, o) \). As
\( \bar{B}_{\tau,P}(F, \blank) \) is uniformly bounded by \( \tau \)
for any \( F \in \mathcal{H}_{\mathcal{P}} \), it follows from section
8.1 in \citep{vaart2006oracle} that
\( (\tau, 1.5 \E_P{[\bar{B}_{\tau,P}(F, O)^2]}) \) is a Bernstein
pair for \( \bar{B}_{\tau,P}(F, \blank) \). Now, for any
\( a,b,c \in \R \) we have
\begin{align*}
  (a-c)^2 - (b-c)^2
  & = (a-b+b-c)^2 - (b-c)^2
  \\
  & = (a-b)^2 + (b-c)^2 +2(b-c)(a-b) - (b-c)^2
  \\
  & = (a-b)
    \left\{
    (a-b) +  2(b-c)
    \right\}
  \\
  & = (a-b)
    \left\{
     a + b -2c
    \right\},
\end{align*}
so using this with \( a=F(t, l, x) \), \( b=F_P(t, l, x) \), and
\( c = \1{\{\eta(t) = l\}} \), we have by Jensen's inequality
\begin{align*}
  & \E_P{[\bar{B}_{\tau,P}(F, O)^2]}
  \\
  & \leq
    2\tau\E_{P}{\left[
    \sum_{l=-1}^{2} \int_0^{\tau}
    \left\{
    \left(
    F(t, l, X) - \1{\{\eta(t) = l\}}
    \right)^2
    -
    \left(
    F_P(t, l, X) - \1{\{\eta(t) = l\}}
    \right)^2
    \right\}^2
    \diff t 
    \right]}
  \\
  & =2\tau
    \E_{P}\Bigg[
    \sum_{l=-1}^{2} \int_0^{\tau}
    \left(
    F(t, l, X) - F_P(t, l, X)
    \right)^2
  \\
  & \quad \quad \quad\quad \quad \quad \times
    \left\{
    F(t, l, X) +  F_P(t, l, X)-2 \1{\{\eta(t) = l\}}
    \right\}^2
    \diff t 
    \Bigg]
  \\
  & \leq
    8\tau \E_{P}{\left[
    \sum_{l=-1}^{2} \int_0^{\tau}
    \left(
    F(t, l, X) - F_P(t, l, X)
    \right)^2
    \diff t 
    \right]}.
  \\
  & =
    8\tau \Vert F - F_P \Vert_{P}^2.
\end{align*}
Thus when \( v(F) = 1.5 \E_P{[\bar{B}_{\tau,P}(F, O)^2]} \) we have by
Lemma~\ref{lemma:norm}
\begin{equation*}
  \frac{v(F)}{R_P(F)}
  = 1.5 \frac{\E_P{[\bar{B}_{\tau,P}(F, O)^2]}}{\E_P{[\bar{B}_{\tau,P}(F, O)]}}
  \leq 12 \tau,
\end{equation*}
and so using the Bernstein pairs \( (\tau, 1.5 \E_P{[\bar{B}_{\tau,P}(F, O)^2]}) \) we have
\begin{equation*}
  \sup_{F \in \mathcal{H}_{\mathcal{P}}}
  \left\{
    M(F) + \frac{v(F)}{R_P(F)}
    \left(
      \frac{1}{\delta} + 1
    \right)
  \right\}
  \leq \tau
  \left(
    13 + \frac{12}{\delta}
  \right).
\end{equation*}
For all $\delta>0$ we thus have
\begin{align*}
  \frac{1}{K} \sum_{k=1}^{K} \E_{P}{\left[ R_P(\hat{\phi}_n(\data_n^{-k})) \right]}
  \leq
  &(1+2\delta)\frac{1}{K} \sum_{k=1}^{K}\E_{P}{\left[ R_P(\tilde{\phi}_n(\data_n^{-k})) \right]}
  \\
  & \quad
    + (1+\delta)\log(1 + |\mathcal{F}_n|) \tau \frac{16 K}{n}
    \left(
    13 + \frac{12}{\delta}
    \right),
\end{align*}
and then the final result follows from Lemma~\ref{lemma:norm}.
\end{proof}

\begin{proof}[of Corollary~\ref{cor:asymp-cons}]
  By definition of the oracle and Lemma~\ref{lemma:norm},
  \begin{equation*}
    \frac{1}{K} \sum_{k=1}^{K} \E_{P}{\left[ \Vert \tilde{\phi}_n(\data_n^{-k}) - F_P \Vert_{P}^2
      \right]} \leq
    \frac{1}{K} \sum_{k=1}^{K}\E_{P}{\left[ \Vert
        \phi_n(\data_n^{-k}) - F_P \Vert_{P}^2
      \right]}
    =
    \E_{P}{\left[ \Vert \phi_n(\data_n^{-k}) - F_P \Vert_{P}^2
      \right]},
  \end{equation*}
  for all \( n \in \N \), where the last equality follows because all
  the training sets \( \data_n^{-k} \) have the same distribution. The
  result then follows from Proposition~\ref{prop:oracle-prop} by
  letting $\delta$ grow to zero with \( n \), for instance as
  $\delta_n = \log(n)^{-\epsilon}$ for some $\epsilon>0$.
\end{proof}






\bibliographystyle{biometrika}
\bibliography{bib.bib}

 




\end{document}\vspace{0.4cm} 
