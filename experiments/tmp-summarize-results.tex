% Created 2023-11-21 Tue 15:20
% Intended LaTeX compiler: pdflatex
\documentclass[a4paper,danish]{article}
\usepackage[utf8]{inputenc}
\usepackage[T1]{fontenc}
\usepackage{graphicx}
\usepackage{longtable}
\usepackage{wrapfig}
\usepackage{rotating}
\usepackage[normalem]{ulem}
\usepackage{amsmath}
\usepackage{amssymb}
\usepackage{capt-of}
\usepackage{hyperref}
\usepackage{listings}
\usepackage{color}
\usepackage{amsmath}
\usepackage{array}
\usepackage[T1]{fontenc}
\usepackage{natbib}
\usepackage[margin=4cm]{geometry}
\usepackage{dsfont, pgfpages, tikz,amssymb, amsmath,xcolor, caption, subcaption}
% Standard library
\usepackage[utf8]{inputenc}
\usepackage[T1]{fontenc}
\usepackage{natbib, dsfont, enumitem,
  amssymb,soul,xcolor,amsmath,graphicx,subcaption,verbatim,pgfplots,tikz,prodint}
\usetikzlibrary{calc,patterns,angles,quotes,automata, positioning,arrows,shapes}

% Ref and bibliography style 
\bibliographystyle{abbrvnat}

% Handling new lines
\setlength{\parskip}{1em}
\setlength{\parindent}{0em}

% Handling space after sections
\usepackage{titlesec}
\titlespacing*{\section}{0em}{2em}{0em}
\titlespacing*{\subsection}{0em}{2em}{0em}
\titlespacing*{\subsubsection}{0em}{2em}{0em}

% No spacing after in start of list
\setlist[itemize]{topsep=0pt}
\setlist[enumerate]{topsep=0pt}

% Todo and notes
% Todo and notes
\usepackage[author=]{fixme}
\fxusetheme{color}
\definecolor{fxtarget}{rgb}{.5,.5,.5}
\definecolor{fxnote}{rgb}{.5,.5,.5}

\newcommand{\darkmode}[1]{ % Set dark mode or not
  \ifthenelse{ \equal{#1}{1}}{
    \usepackage{pagecolor}
    \definecolor{pagecol}{RGB}{0,43,54}
    \definecolor{textcol}{RGB}{131,148,150}
    % \definecolor{textcol}{RGB}{38,139,210}
    \pagecolor{pagecol}
    \color{textcol}
  }
}
% New operators and commands
\newcommand{\R}{\mathbb{R}}
\newcommand{\N}{\mathbb{N}}
\newcommand{\blank}{\makebox[1ex]{\textbf{$\cdot$}}}
\newcommand\independent{\protect\mathpalette{\protect\independenT}{\perp}}
\def\independenT#1#2{\mathrel{\rlap{$#1#2$}\mkern2mu{#1#2}}}
\renewcommand{\phi}{\varphi}
\renewcommand{\epsilon}{\varepsilon}
\newcommand*\diff{\mathop{}\!\mathrm{d}}
\newcommand{\weakly}{\rightsquigarrow}
\newcommand\smallO{\textit{o}}
\newcommand\bigO{\textit{O}}
\newcommand{\midd}{\; \middle|\;}
\newcommand{\1}{\mathds{1}}
\usepackage{ifthen} %% Empirical process with default argument
\newcommand{\G}[2][n]{
{\ensuremath{\mathbb{G}_{#1}}{\left[#2\right]}}
}
\DeclareMathOperator*{\argmin}{\arg\!\min}
\DeclareMathOperator*{\argmax}{\arg\!\max}
\newcommand{\V}{\mathrm{Var}} % variance
\newcommand{\eqd}{\stackrel{d}{=}} % equality in distribution
\newcommand{\arrow}[1]{\xrightarrow{\; {#1} \;}}
\newcommand{\arrowP}{\xrightarrow{\; P \;}} % convergence in probability
\newcommand{\KL}{\ensuremath{D_{\mathrm{KL}}}}

%% Easily change notation
\newcommand{\leb}{\lambda} % the Lebesgue measure
\DeclareMathOperator{\TT}{\Psi} % target parameter
\newcommand{\lp}{\ensuremath{\mathcal{L}_{P}^2}} % shortcut for lp2 space
\newcommand{\empmeas}{\ensuremath{\mathbb{P}_n}} % empirical measure
\newcommand{\E}{{\ensuremath{\mathop{{\mathbb{E}}}}}} % expectation
\newcommand{\ic}{\ensuremath{I}} % influence curve
% Steeling style from org beamer:
\lstset{
   keywordstyle=\color{blue},
   commentstyle=\color{red},stringstyle=\color[rgb]{0,.5,0},
   literate={~}{$\sim$}{1},
   basicstyle=\ttfamily\small,
   columns=fullflexible,
   breaklines=true,
   breakatwhitespace=false,
   numbers=left,
   numberstyle=\ttfamily\tiny\color{gray},
   stepnumber=1,
   numbersep=10pt,
   backgroundcolor=\color{white},
   tabsize=4,
   keepspaces=true,
   showspaces=false,
   showstringspaces=false,
   xleftmargin=.23in,
   frame=single,
   basewidth={0.5em,0.4em},
}
\usepackage{transparent}
\makeatletter         
\renewcommand\maketitle{  
  {\raggedright
        \begin{flushright}
          % {\color{white} 1} \\[-2cm] \@date % ... very hackish solution...
          {\color{white}\transparent{0} 1} \\[-2cm] \@date % ... very hackish solution...
    \end{flushright}
    \begin{center}          
      {\LARGE  \@title \par}
      \@author\\[6ex]
    \end{center}
}}
\makeatother
\author{Anders Munch}
\date{\today}
\title{Summary of some simulation results}
\begin{document}

\maketitle
\section{Super learners}
\label{sec:orgfb13ca2}
In all simulation studies, we compare five super learners, which are listed
below. To evaluate performance super learner, we use an independent data set of
10.000 uncensored samples and calculate the integrated Brier score in this data
set. All results are based on 500 simulated data sets.

\begin{enumerate}
\item The state learner (referred to as \texttt{statelearner}).
\item The super learner proposed in \citep{westling2021inference} (referred to as
\texttt{survSL}).
\item A super learner based on the estimated integrated Brier score, where the
censoring mechanism is estimated with the Kaplan-Meier estimator (referred to
as \texttt{ipcw\_km}).
\item A super learner based on the estimated integrated Brier score, where the
censoring mechanism is estimated with a Cox model that includes all available
covariates as main effects (referred to as \texttt{ipcw\_cox}).
\item The (discrete) oracle super learner, which picks the model that minimizes the
Brier score in the independent data of 10.000 uncensored samples (referred to
as \texttt{oracle}).
\end{enumerate}

Only the super learners 1., 2., and 3. provides estimates of the censoring distribution.

\section{Zelefsky based simulation}
\label{sec:org1123149}
We generate data in four different ways:

\begin{enumerate}
\item Data as generated in \citep{gerds2013estimating} (referred to as \texttt{original}).
\item As in 1., but where censoring in completely independent of covariates
(referred to as \texttt{indep\_cens}).
\item Same censoring mechanism as in 1., but where the outcome depend only on one
of the covariates (referred to as \texttt{simple\_effect})
\item As in 1., but we add 5 independent standard Guassian covariates with no
effect on neither outcome nor censoring (referred to as \texttt{noise}).
\end{enumerate}

\clearpage

\subsection{Kaplan-Meier, Cox, and random forest}
\label{sec:org0516d66}
In this setting, we include the following learners in all libraries:

\begin{itemize}
\item The Kaplan-Meier estimator
\item A Cox model with main effects
\item A random forest based on 50 trees
\end{itemize}


\begin{figure}[htbp]
\centering
\includegraphics[width=1\linewidth]{/tmp/babel-IUB42s/figure-EtEIk3.pdf}
Zelefsky simulation setting using library consisting of Kaplan-Meier, Cox model, and random forests
\end{figure}

\clearpage

\subsection{Add LASSO}
\label{sec:orgd0d0620}
In this setting we add a learner to all libraries, so that all libraries include
the learners:

\begin{itemize}
\item The Kaplan-Meier estimator
\item A Cox model with main effects
\item A random forest based on 50 trees
\item A penalized Cox model with main effects, where the \(\|\blank\|_1\) penalty
(LASSO) is used and the penalty parameter is selected using cross-validation
based on Cox' partial likelihood
\end{itemize}

\begin{figure}[htbp]
\centering
\includegraphics[width=1\linewidth]{/tmp/babel-IUB42s/figure-KMJCXV.pdf}
Zelefsky simulation setting including LASSO into the library
\end{figure}

\clearpage


\section{Effect of number of variables}
\label{sec:orgf46a6e0}
We generate data in three different way:

\begin{enumerate}
\item Outcome and censoring depends on one binary covariate (\(X_1\)). Another
continuous covariate (\(X_2\)) that is correlated with \(X_1\) is generated.
\item Same as in 1., but we also add 4 independent Gaussian covariates (\(X_3,
   \dots X_6\)).
\item Same as in 1., but we also add 9 independent Gaussian covariates (\(X_3,
   \dots X_{11}\)).
\end{enumerate}

\subsection{Kaplan-Meier, Cox, and random forest}
\label{sec:org4b465da}
In this setting, we include the following learners in all libraries:

\begin{itemize}
\item The Kaplan-Meier estimator
\item A Cox model with main effects
\item A random forest based on 50 trees
\end{itemize}

\begin{figure}[htbp]
\centering
\includegraphics[width=1\linewidth]{/tmp/babel-IUB42s/figure-TImPsh.pdf}
Zelefsky simulation setting using library consisting of Kaplan-Meier, Cox model, and random forests
\end{figure}

\clearpage

\subsection{Add LASSO}
\label{sec:org79abb0b}
In this setting we add a learner to all libraries, so that all libraries include
the learners:

\begin{itemize}
\item The Kaplan-Meier estimator
\item A Cox model with main effects
\item A random forest based on 50 trees
\item A penalized Cox model with main effects, where the \(\|\blank\|_1\) penalty
(LASSO) is used and the penalty parameter is selected using cross-validation
based on Cox' partial likelihood
\end{itemize}

\begin{figure}[htbp]
\centering
\includegraphics[width=1\linewidth]{/tmp/babel-IUB42s/figure-ozs9vG.pdf}
Zelefsky simulation setting including LASSO into the library
\end{figure}



\section{References}
\label{sec:org53b63a4}
\renewcommand{\section}[2]{} 
\bibliography{./latex-settings/default-bib.bib}
\end{document}